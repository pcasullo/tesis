\chapter{Lecciones aprendidas}

\section{Aspectos que funcionaron bien}

Durante el desarrollo y la implementación del proyecto ReportHub, se identificaron varios aspectos que funcionaron de manera efectiva. En primer lugar, la colaboración entre los diferentes equipos de la organización fue uno de los mayores éxitos. La comunicación fluida y el trabajo en equipo permitieron resolver problemas de manera rápida y eficiente, asegurando que el proyecto avanzara según lo planeado. Además, la metodología ágil utilizada permitió una entrega iterativa y continua, lo que facilitó la incorporación de \textit{feedback} y la realización de ajustes en cada fase del proyecto. La arquitectura orientada a servicios (SOA) también demostró ser una elección acertada, ya que permitió una escalabilidad y flexibilidad significativas en el desarrollo y la implementación de la solución.

\section{Dificultades y cómo se superaron}

A lo largo del proyecto, se enfrentaron varias dificultades que requirieron soluciones creativas y eficaces. Una de las principales dificultades fue la evolución de la metodología de desarrollo, donde quedó claro, que con el enfoque inicial, no se hubiese podido avanzar ni con la velocidad, ni con la calidad necesaria. 

\section{Factores clave de éxito}

Varios factores clave contribuyeron al éxito del proyecto ReportHub. En primer lugar, el compromiso y la dedicación del equipo de proyecto fueron fundamentales. Desde el inicio, todos los miembros del equipo estuvieron alineados con los objetivos del proyecto y trabajaron de manera colaborativa para alcanzarlos. La metodología ágil también jugó un papel crucial, permitiendo una entrega iterativa y la incorporación de \textit{feedback} continuo. Además, la elección de una arquitectura escalable y auditable permitió que la solución se adaptara a las necesidades cambiantes de la organización y facilitó el cumplimiento de normativas y la mejora de la seguridad. Por último, el apoyo y la participación de los \textit{stakeholders} fueron vitales para asegurar la alineación del proyecto con los objetivos estratégicos de la organización.

\section{Aspectos a mejorar}

A pesar de los éxitos alcanzados, se identificaron varios aspectos que podrían mejorarse en futuros proyectos. Uno de estos aspectos es la gestión del cambio. Aunque se llevaron a cabo sesiones de capacitación y comunicación, algunos usuarios aún mostraron resistencia al cambio. En futuros proyectos, se podría considerar la implementación de un plan de gestión del cambio más robusto, que incluya una comunicación más temprana y continua, así como una mayor participación de los usuarios en las etapas iniciales del proyecto. Otro aspecto a mejorar es la documentación. Aunque se creó documentación detallada, algunos usuarios encontraron que esta no era lo suficientemente clara o accesible. Mejorar la claridad y la accesibilidad de la documentación podría facilitar una adopción más rápida y eficiente de la nueva herramienta.

\section{Aprendizajes organizacionales}

El proyecto ReportHub proporcionó varios aprendizajes valiosos a nivel organizacional. Uno de los principales aprendizajes fue la importancia de la colaboración y la comunicación efectiva entre los diferentes equipos y departamentos. Este proyecto demostró que el trabajo en equipo y la comunicación fluida son esenciales para el éxito de cualquier iniciativa. Además, se aprendió la importancia de la flexibilidad y la adaptabilidad. La metodología ágil y la arquitectura escalable permitieron que el proyecto se adaptara a los cambios y necesidades emergentes de la organización, lo que fue crucial para su éxito. También se destacó la importancia de involucrar a los \textit{stakeholders} desde el inicio del proyecto para asegurar su alineación con los objetivos estratégicos de la organización.

\section{Recomendaciones para futuros proyectos}

Basándonos en las lecciones aprendidas del proyecto ReportHub, se pueden hacer varias recomendaciones para futuros proyectos. En primer lugar, es esencial implementar un plan de gestión del cambio robusto que incluya una comunicación temprana y continua, así como la participación de los usuarios en las etapas iniciales del proyecto. También se recomienda mejorar la claridad y la accesibilidad de la documentación para facilitar una adopción más rápida y eficiente de las nuevas herramientas. Además, es importante mantener un enfoque colaborativo y asegurar una comunicación efectiva entre los diferentes equipos y departamentos de la organización. Finalmente, se sugiere seguir utilizando metodologías ágiles y arquitecturas escalables para asegurar la flexibilidad y la adaptabilidad de los proyectos a las necesidades cambiantes de la organización.