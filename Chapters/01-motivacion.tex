\chapter{Motivacion}

\section{Contexto organizacional}
%{\begin{small}%
%\begin{flushright}%
%\it
%There's nothing for me now.
%I want to learn the ways of\\ the Force and become a Jedi like my father. \\
%--Luke Skywalker
%\end{flushright}%
%\end{small}%
%\vspace{.5cm}}

En las grandes corporaciones multinacionales, existen múltiples tecnologías para resolver distintos desafíos de análisis e inteligencia de negocios, a lo largo de todo el stack tecnológico que va desde los sistemas \gls{etl} \cite{inmon1992}, distintos repositorios de almacenamiento para datos estructurados y no estructurados, y herramientas de exploración y explotación de datos que se asocian a la capa de presentación. En esta última coexisten planillas de cálculo (xls) y software específico de visualización para crear reportes y tableros de información, tales como Microstrategy, Cognos, PowerBI, Tableau o Qlik, entre otros.

Estas corporaciones suelen definir estándares tecnológicos que determinan herramientas o proveedores oficiales, con el fin de simplificar y lograr sinergias entre las tecnologías seleccionadas y reducir las combinaciones según distintos casos de uso \cite{jacobson1986}.

Adicionalmente, es común que las iniciativas de inteligencia de negocios estén descentralizadas y distribuidas tanto en áreas funcionales (marketing, ventas, compras, finanzas), como en distintos niveles geográficos, que abarcan desde equipos corporativos hasta instancias globales, regionales y locales.

\section{Situación inicial y limitaciones detectadas}

Este contexto de distribución de recursos en áreas y geografías, combinado con la multiplicidad de herramientas “aprobadas” por dichas corporaciones, genera una combinación de componentes para la creación de distintas soluciones que, aun siguiendo las mejores prácticas de gobernanza tecnológica y de arquitectura \cite{isaca2012}, devienen en un ecosistema complejo, desarmonizado y lleno de particularidades.

Poder establecer el valor real e impacto que estas soluciones tienen en el negocio es un desafío, ya que en muchos casos la relación entre adopción e impacto es indirecta y no trivial \cite{davenport2006}. A su vez, la experiencia de los distintos usuarios varía significativamente: solicitar acceso o encontrar los enlaces de cada herramienta resulta inconsistente, incoherente y dependiente de qué equipo haya desarrollado la solución.

\section{Problemas principales para usuarios y equipos de datos}
\subsection{Gestión de accesos}

La gestión de accesos junto con la aplicación de políticas de seguridad una vez otorgados los mismos, variaban absolutamente y quedaban definidas por el criterio de cada equipo que desarrollaba los contenidos, sin haber coherencia alguna.
Algunos ejemplos de variantes incluyen:

Enviar un email a quien dio las capacitaciones (en caso que las haya habido). Muchas veces la persona encargada de la instrucción o entrenamiento de usuarios era también la encargada de controlar el acceso y otorgarlo. En otros casos, actuaba como intermediario para poder llegar a quien era responsable de otorgar los accesos adecuados.
Solicitar acceso mediante herramientas de tickets a equipos encargados de la operación de dichos tableros/reportes. En equipos de proyecto de tamaños medio o grandes (con un número de integrantes excediendo la decena), es común que haya gente dedicada exclusivamente a tareas operativas, cuyo objetivo es por un lado garantizar la continuidad y máxima disponibilidad de las soluciones como así también en ocasiones se encargan de ejecutar las actividades que habilitan acceso y perfiles de seguridad adecuados a los usuarios.
Solicitar acceso a alguien conocido si puede hacer de nexo para dar con el contacto indicado.

Finalmente, por el diseño (o falta de diseño) de estos procesos, hay múltiples intermediarios en dichos pedidos y otorgamientos, cuando en realidad, en escencia, deberían existir idealmente dos roles:

\begin {itemize}
\item quien solicita acceso.
\item quien aprueba, define y otorga (en una sola persona). 
\end {itemize}

Por estas complejidades accidentales, muchas veces, quien lo aprueba y define, no es quien se encarga de que se haga el otorgamiento efectivo. Y es aquí donde aparece una de las mayores oportunidades de simplificación y opimización.

\subsection{Consumo de contenidos}

Para poder consumir o utilizar reportes es fundamental saber de su existencia, conocer su ubicación (normalmente son enlaces dentro de redes internas) y tener acceso a los mismos.
Algunos ejemplos acerca de cómo acceder, pueden incluir:

\begin{enumerate}
    \item Usuarios reciben enlaces de acceso por emails o en documentos de entrenamientos y luego (con suerte) los almacenan como atajos en sus navegadores (con los problemas de tener hardcoded links que luego con el tiempo pueden variar, apuntar a versiones obsoletas de dichos reportes o incluso quedar deprecados conforme algunos reportes son decomisionados).
    \item Páginas de intranet donde se publican los puntos de acceso (muchas veces implementadas como portales de acceso, provistos por las mismas herramientas de visualización).
    \item Usuarios reciben reportes como archivos adjuntos en correos electrónicos.
\end{enumerate}

\subsection{Manejo del portafolio de soluciones de inteligencia de negocios}

Para poder hacer un manejo efectivo de un portafolio de soluciones y de las tecnologías que se utilizan, es necesario poder entender:

¿Qué soluciones hay disponibles y en qué tecnologías están desarrolladas (inventario)?

¿Qué áreas de negocio cubren y quienes son responsables?

¿Qué nivel de adopción tienen?

¿Qué usuarios tienen acceso y no deberían?

¿Qué usuarios necesitan acceso y no tienen?

¿Qué usuarios tienen acceso y no lo usan?

¿Qué usuarios tienen alto nivel de adopción?

¿Qué volumen de pedidos de acceso manejan, según las audiencias objetivo?

¿Hay duplicidad de contenidos, hechos por áreas afines pero sin colaborar?

¿Hay patrones de uso de reportes que tengan correlación con el desempeño de áreas de negocio?

La respuesta a cada una y en particular a todas estas preguntas, en el entorno descripto, es de un esfuerzo que no permite tener información en tiempo y forma ni de modo adecuado, repetible y consistente de manera constante. Poder responder cada pregunta en un contexto tan heterogéneo, implica un nivel de trabajo manual, armonización y alineación de criterios y definiciones que simplemente lo vuelven imposible, en la escala de estas organizaciones.

\newpage
En resumen, los impactos negativos descriptos anteriormente pueden sintetizarse en la siguiente lista: 

\begin{itemize}
\item Falta de un repositorio central y común donde buscar/encontrar y solicitar acceso a los contenidos necesarios, genera dificultades para poder dar con los contenidos.
\item Procesos inconsistentes y altamente variables.
\item Falta de transparencia en cuanto a puntos de contactos (personas responsables) de reportes.
\item Métodos de aprobación de múltiples pasos y manuales, que dependen de horarios laborales, feriados distribuidos en varias husos horarios y múltiples geografías.
\item Separación entre nivel de aprobación (hecha por la persona responsable) de un acceso y el nivel de ejecución de dicho acceso (hecha por operadores, una vez que la persona responsable ha definido qué acceso corresponde).
\item Falta de información necesaria para el aprobador, acerca de rol, país, función y otros elementos que ayudan a determinar si un acceso debe o no ser otorgado a un usuario.
\item La ausencia de un repositorio centralizado agrava los problemas de consistencia y coherencia. Esto se traduce en dificultades para asegurar estándares de seguridad homogéneos, en la duplicación de esfuerzos.
\item El estado fragmentado limita la capacidad de contar con indicadores confiables y oportunos para la toma de decisiones de portafolio. 
\item La falta de métricas unificadas de adopción impide evaluar el impacto de las soluciones desarrolladas, dificultando la gestión adecuada del portafolio de inteligencia de negocios.
\end{itemize}


\section{Aporte de la Tesis}
Esta \textit{tesis}, en formato de Experience Report, tiene como objetivo, profundizar cómo fue el proceso de desarrollo de una solución que atendió a los desafíos mencionados, detallando desde su concepción en el contexto organizacional, hasta su despliegue y medición del impacto, dentro de una organización de estas características.

Los contenidos a desarrollar incluyen:

\textbf{Definiciones Preliminares:} Conjunto de definiciones básicas que nos introducen en el dominio del problema.

\textbf{Propuesta:} Descripción detallada conceptual de la solución, en función de sus requerimientos funcionales y no funcionales. \cite{zave1979} \cite{yeh1980}

\textbf{Arquitectura:} Descripción de los componentes y sus relaciones como también algunos patrones de diseño aplicados. \cite{yourdon1979} \cite{erl2005} \cite{fowler2003}

\textbf{Desarrollo y Despliegue:} Explicación de la metodología aplicada y entregas iterativas. 

\textbf{Evaluación:} Monitoreo y evaluación de las métricas clave y factores de éxito.

\textbf{Lecciones aprendidas:} Hallazgos y aprendizajes de la experiencia en el proyecto.

\textbf{Conclusiones:} Síntesis final del trabajo, con implicancias finales.

%\unsure { es una prueba de unsure para ver si funciona, está piola}
