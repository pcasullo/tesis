\chapter{desarrollo y despliegue}
\section{Enfoque metodológico (Waterfall/Agile)}

Metodología Waterfall
En la metodología Waterfall, las fases del desarrollo de software se llevan a cabo de manera secuencial. Aquí tienes cómo se organiza el equipo en esta metodología:

Product Owner: Responsable de definir los requerimientos y asegurarse de que el producto final cumpla con las expectativas del cliente.
Scrum Master: Este rol no existe en Waterfall, pero podría haber un jefe de proyecto que coordina las tareas y asegura que las fases se completen a tiempo.
Programadores de Front End: Se encargan de la interfaz de usuario una vez que se ha definido y aprobado el diseño.
Programadores de Back End y de Base de Datos: Trabajan en la lógica del servidor y la estructura de la base de datos después de la fase de diseño.
Testers: Realizan pruebas finales después de que el desarrollo esté completo.
Operadores: Se encargan del despliegue y mantenimiento del sistema una vez que ha sido probado y aprobado.
Metodología Agile
En la metodología Agile, las fases del desarrollo de software son iterativas e incrementales. El equipo trabaja en sprints cortos y frecuentes para entregar funcionalidades completas. Aquí es cómo se organiza el equipo:

Product Owner: Define y prioriza el backlog del producto, asegurándose de que el equipo trabaje en las tareas de mayor valor.
Scrum Master: Facilita el proceso Agile y remueve obstáculos para el equipo.
Programadores de Front End: Trabajan en la interfaz de usuario en cada sprint, colaborando estrechamente con los programadores de back end.
Programadores de Back End y de Base de Datos: Desarrollan la lógica del servidor y la estructura de la base de datos en cada sprint.
Testers: Realizan pruebas continuas durante cada sprint, asegurando que las nuevas funcionalidades funcionen correctamente.
Operadores: Empiezan a involucrarse en el proceso de despliegue más frecuentemente, aunque aún puede haber una separación entre desarrollo y operaciones.

\section{Planificación y backlog inicial}

\section{Definición de historias de usuario clave}

\section{Iteraciones y sprints principales}

\section{Validaciones con usuarios de negocio}

\section{Gestión de calidad}

\subsection{Pruebas funcionales}

\subsection{Pruebas de seguridad}
\subsection{Pruebas de performance}
\subsection{Pruebas de integración}
\section{Hitos alcanzados y ajustes sobre la marcha}
\section{Estrategia de despliegue y fases de rollout.}
\section{Piloto inicial y resultados}
\section{Escalamiento a más áreas y usuarios}
\section{Plan de comunicación y capacitación}
\section{Materiales de soporte y guías de usuario}
\section{Gestión del cambio organizacional}
\section{Mecanismos de soporte post-lanzamiento}

