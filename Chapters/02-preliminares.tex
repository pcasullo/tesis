\chapter{Definiciones Preliminares}

Las definiciones preliminares son aquellas que nos permiten profundizar en los elementos básicos del dominio de nuestro contexto. A continuación se presentan las que aplican a conceptos, roles y elementos clave utilizados a lo largo de este trabajo.

\section{Definiciones de industria y corporaciones multinacionales}
\begin{description}
    \item [KPIs:] Key Performance Indicators (Indicadores clave de desempeño).
\end{description}

\section{Definiciones de inteligencia de negocios}

\begin{description}
    \item [Inteligencia de negocios:] (Business intelligence)
    \item [Data Warehouse:]
\end{description}

\section{Definiciones de diseño y arquitectura de software}
\begin{description}
    \item[Stack tecnológico:] 
    \item [Web App:]
    \item [RLS:]
    \item [LDAP:]
    \item [API:]
    \item [SOA:]
    \item [Cohesión:]
    \item [Acoplamiento:]
    \item [Web Server:]
    \item [Load Balancer:]
    \item [Web Application Server:]
    \item [Back end:]
    \item [Client Server:]
    \item [Caché:]
    \item [Tolerancia a fallas:]
    \item [NFS:]
    \item [Database Server:]
    \item [MVC:] Model View Controller \cite{pope1988}. Patrón de diseño mediante el cual, se separan en tres capas bien definidas, para mayor flexibilidad, mantenibilidad y escalabilidad.
    
\end{description}

\section{Definiciones de metodologías}
    
\begin{description}

\item[Waterfall/Cascada:] Descripción de la Metodología waterfall.
\item[Agile/Agil:] Descripción de la Metodología ágil.
\item[DevOps:] Metodología DevOps.
\item[Unit tests:] Pruebas unitarias.
\item[Integration tests:] Pruebas de integración.
\item[Dev:] Desarrollo. Entorno para desarrollo y pruebas de unidad y de integración.
\item[UAT:] User Acceptance Test. Entorno de pruebas de usuario y validaciones finales antes de hacer pasajes a producción.
\item[Pr:] Production/Producción. Entorno de operaciones productivo.

\end{description}

\section{Definiciones propias del proyecto}

\begin{description}

\item[ReportHub:] Plataforma propuesta para consolidar y armonizar la publicación de reportes y tableros, gestión de accesos y monitoreo de métricas de uso de BI.

\item[Administradores:] Usuarios con privilegios para gestionar contenidos, accesos y usuarios dentro de la plataforma. Se distinguen tres tipos:
    \begin{itemize}
        \item \textbf{Admin Global:} Gestiona unidades geográficas globales y áreas de información a nivel global.
        \item \textbf{Admin Local:} Gestiona áreas de información dentro de su región geográfica asignada.
    \end{itemize}

\item[Creadores de Contenidos:] Usuarios responsables de generar y publicar contenidos dentro de la plataforma.

\item[Consumidores de Contenidos:] Usuarios que acceden y utilizan los contenidos publicados en el portal, pudiendo también marcar favoritos y realizar solicitudes de acceso.

\item[Aprobadores:] Usuarios que evalúan y aprueban solicitudes de acceso a contenidos según la configuración de seguridad definida.

\item[Estructura de contenidos:] Jerarquía de organización de contenidos en dos niveles:
    \begin{enumerate}
        \item \textbf{Nivel 1:} Unidades geográficas (Global, Europa, América, Asia, África y países correspondientes).
        \item \textbf{Nivel 2:} Áreas de información (temáticas específicas, únicas dentro de cada región).
    \end{enumerate}

\item[Metadatos de Contenido:] Información asociada a cada contenido publicado, incluyendo:
    \begin{itemize}
        \item Título
        \item Descripción
        \item Imagen miniatura
        \item Tipo de contenido (archivo o URL)
        \item Segmentos de usuarios y objetivos de frecuencia de uso
        \item Aprobadores
        \item Configuración de provisión de acceso
    \end{itemize}

\end{description}

\section{Definiciones de diseño y arquitectura de software}

\begin{description}

    \item [Stack tecnológico:] Conjunto de tecnologías y herramientas utilizadas para construir y ejecutar una aplicación o sistema \gls{techstack}.
    \item [Web App:] Aplicación que se ejecuta en un navegador web y accede a servicios a través de Internet \gls{wapp}.
    \item [RLS:] \gls{rls}, directrices o políticas que definen cómo se maneja y protege la información sensible.
    \item [LDAP:] \gls{ldap}, un protocolo de red para acceder a servicios de directorio de información.
    \item [API:] \gls{api}, un conjunto de rutinas y protocolos de programación que permiten la creación de software.
    \item [SOA:] \gls{soa}, un enfoque de diseño en el que tanto las aplicaciones como los componentes de las aplicaciones son services intercambiables.
    \item [Cohesión:] La medida en que las funciones relacionadas se agrupan en un módulo o clase \cite{yourdon1979}.
    \item [Acoplamiento:] La dependencia entre módulos o componentes de software \cite{yourdon1979}.
    \item [\Gls{webserver}:] Servidor que sirve páginas web y gestiona las solicitudes HTTP.
    \item [\Gls{loadbalancer}:] Dispositivo de red que distribuye la carga de trabajo entre múltiples servidores.
    \item [Web Application Server:] Servidor que gestiona aplicaciones web y proporciona servicios adicionales como escalado y seguridad.
    \item [Back end:] Parte del sistema que maneja la lógica de negocio, el almacenamiento de datos y la interacción con el front end \gls{backend}.
    \item [\Gls{clientserver}:] Modelo de arquitectura de red en el que los clientes solicitan servicios al servidor.
    \item [\Gls{cache}:] Componente de hardware o software que almacena copias temporales de datos para acceder a ellos más rápidamente.
    \item [Tolerancia a fallas:] La capacidad de un sistema para continuar funcionando a pesar de errores o fallos.
    \item [NFS:] \Gls{nfs}, sistema de archivos central y compartido que permite a múltiples computadoras acceder a archivos de forma simultánea.
    \item [\Gls{dbserver}:] Un servidor que gestiona bases de datos y permite el acceso y la manipulación de datos.
    \item [MVC:] Model View Controller \cite{pope1988}. Patrón de diseño mediante el cual, se separan en tres capas bien definidas, para mayor flexibilidad, mantenibilidad y escalabilidad.
    
\end{description}

\section{Definiciones de metodologías}
    
\begin{description}

\item[Waterfall/Cascada:] Una metodología de desarrollo de software en la que el proceso se divide en fases secuenciales y se asume que cada fase se completa antes de pasar a la siguiente \cite{royce1970}.
\item[Agile/Agil:] Una metodología de desarrollo de software que promueve la iteración y la colaboración, y que se adapta a los cambios a medida que el proyecto avanza.% \cite{agile2001}.
\item[DevOps:] El término "DevOps" fue acuñado por Patrick Debois en 2009 para nombrar su conferencia "DevOpsDays", la cual se inspiró en la charla "10 despliegues por día" de John Allspaw y Paul Hammond en la Conferencia Velocity de 2009. El término se creó al combinar Development (desarrollo) y Ops (operaciones) para resolver la brecha entre los dos equipos.\cite{allspaw2009}.
\item[Unit tests:] Pruebas unitarias.
\item[Integration tests:] Pruebas de integración.
\item[Dev:] Desarrollo. Entorno para desarrollo y pruebas de unidad y de integración.
\item[UAT:] User Acceptance Test. Entorno de pruebas de usuario y validaciones finales antes de hacer pasajes a producción.
\item[Pr:] Production/Producción. Entorno de operaciones productivo.

\end{description}

\section{Definiciones propias del proyecto}

\begin{description}

\item[ReportHub:] Plataforma propuesta para consolidar y armonizar la publicación de reportes y tableros, gestión de accesos y monitoreo de métricas de uso de BI.

\item[Administradores:] Usuarios con privilegios para gestionar contenidos, accesos y usuarios dentro de la plataforma. Se distinguen tres tipos:
    \begin{itemize}
        \item \textbf{Admin Global:} Gestiona unidades geográficas globales y áreas de información a nivel global.
        \item \textbf{Admin Local:} Gestiona áreas de información dentro de su región geográfica asignada.
    \end{itemize}

\item[Creadores de Contenidos:] Usuarios responsables de generar y publicar contenidos dentro de la plataforma.

\item[Consumidores de Contenidos:] Usuarios que acceden y utilizan los contenidos publicados en el portal, pudiendo también marcar favoritos y realizar solicitudes de acceso.

\item[Aprobadores:] Usuarios que evalúan y aprueban solicitudes de acceso a contenidos según la configuración de seguridad definida.

\item[Estructura de contenidos:] Jerarquía de organización de contenidos en dos niveles:
    \begin{enumerate}
        \item \textbf{Nivel 1:} Unidades geográficas (Global, Europa, América, Asia, África y países correspondientes).
        \item \textbf{Nivel 2:} Áreas de información (temáticas específicas, únicas dentro de cada región).
    \end{enumerate}

\item[Metadatos de Contenido:] Información asociada a cada contenido publicado, incluyendo:
    \begin{itemize}
        \item Título
        \item Descripción
        \item Imagen miniatura
        \item Tipo de contenido (archivo o URL)
        \item Segmentos de usuarios y objetivos de frecuencia de uso
        \item Aprobadores
        \item Configuración de provisión de acceso
    \end{itemize}

\end{description}

