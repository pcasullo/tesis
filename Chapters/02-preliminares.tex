\chapter{Definiciones Preliminares}

Las definiciones preliminares son aquellas que nos permiten profundizar en los elementos básicos del dominio de nuestro contexto. A continuación se presentan las que aplican a conceptos, roles y elementos clave utilizados a lo largo de este trabajo.

\section{Definiciones de industria y corporaciones multinacionales}
\begin{description}
    \item [KPIs:] Key Performance Indicators (Indicadores clave de desempeño).
\end{description}

\section{Definiciones de inteligencia de negocios}

\begin{description}
    \item [Inteligencia de negocios:] (Business intelligence)
    \item [Data Warehouse:]
\end{description}

\section{Definiciones de diseño y arquitectura de software}
\begin{description}
    \item[Stack tecnológico:] 
    \item [Web App:]
    \item [RLS:]
    \item [LDAP:]
    \item [API:]
    \item [SOA:]
    \item [Cohesión:]
    \item [Acoplamiento:]
    \item [Web Server:]
    \item [Load Balancer:]
    \item [Web Application Server:]
    \item [Back end:]
    \item [Client Server:]
    \item [Caché:]
    \item [Tolerancia a fallas:]
    \item [NFS:]
    \item [Database Server:]
    \item [MVC:] Model View Controller \cite{pope1988}. Patrón de diseño mediante el cual, se separan en tres capas bien definidas, para mayor flexibilidad, mantenibilidad y escalabilidad.
    
\end{description}

\section{Definiciones de metodologías}
    
\begin{description}

\item[Waterfall/Cascada:] Descripción de la Metodología waterfall.
\item[Agile/Agil:] Descripción de la Metodología ágil.
\item[DevOps:] Metodología DevOps.
\item[Unit tests:] Pruebas unitarias.
\item[Integration tests:] Pruebas de integración.
\item[Dev:] Desarrollo. Entorno para desarrollo y pruebas de unidad y de integración.
\item[UAT:] User Acceptance Test. Entorno de pruebas de usuario y validaciones finales antes de hacer pasajes a producción.
\item[Pr:] Production/Producción. Entorno de operaciones productivo.

\end{description}

\section{Definiciones propias del proyecto}

\begin{description}

\item[ReportHub:] Plataforma propuesta para consolidar y armonizar la publicación de reportes y tableros, gestión de accesos y monitoreo de métricas de uso de BI.

\item[Administradores:] Usuarios con privilegios para gestionar contenidos, accesos y usuarios dentro de la plataforma. Se distinguen tres tipos:
    \begin{itemize}
        \item \textbf{Admin Global:} Gestiona unidades geográficas globales y áreas de información a nivel global.
        \item \textbf{Admin Local:} Gestiona áreas de información dentro de su región geográfica asignada.
    \end{itemize}

\item[Creadores de Contenidos:] Usuarios responsables de generar y publicar contenidos dentro de la plataforma.

\item[Consumidores de Contenidos:] Usuarios que acceden y utilizan los contenidos publicados en el portal, pudiendo también marcar favoritos y realizar solicitudes de acceso.

\item[Aprobadores:] Usuarios que evalúan y aprueban solicitudes de acceso a contenidos según la configuración de seguridad definida.

\item[Estructura de contenidos:] Jerarquía de organización de contenidos en dos niveles:
    \begin{enumerate}
        \item \textbf{Nivel 1:} Unidades geográficas (Global, Europa, América, Asia, África y países correspondientes).
        \item \textbf{Nivel 2:} Áreas de información (temáticas específicas, únicas dentro de cada región).
    \end{enumerate}

\item[Metadatos de Contenido:] Información asociada a cada contenido publicado, incluyendo:
    \begin{itemize}
        \item Título
        \item Descripción
        \item Imagen miniatura
        \item Tipo de contenido (archivo o URL)
        \item Segmentos de usuarios y objetivos de frecuencia de uso
        \item Aprobadores
        \item Configuración de provisión de acceso
    \end{itemize}


\end{description}
