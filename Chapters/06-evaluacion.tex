\chapter{evaluacion}

Medir el éxito de este proyecto fue fundamental por varias razones:

\begin{enumerate}
    \item Evaluación del rendimiento: Pudimos evaluar si los objetivos del proyecto se cumplieron y si los resultados esperados se alcanzaron. Esto ayudó a determinar si el proyecto fue efectivo y eficiente.
    \item Identificación de áreas de mejora: Al medir el éxito, tambien pudimos identificar áreas que funcionaron bien y las que necesitan mejoras. Esto fue vital para aprender de la experiencia y aplicar esos aprendizajes a futuros proyectos.
    \item Justificación de inversiones: Este proyecto requirió de una inversión, tanto en términos de tiempo como de dinero. La evaluación del éxito ayudó a justificar esta inversión, a demostrar el retorno sobre la inversión (ROI) a los interesados y sponsors y habilitar futuras fases de financiamiento para su evolución en el tiempo.
    \item Motivación y reconocimiento: El haber medido y celebrado el impacto del proyecto, motivó al equipo del proyecto, reconociendo sus esfuerzos y logros. Esto mejoró la moral e imagen del equipo en la compañía y ayudó a fomentar una cultura de éxito y colaboración.
    \item Mejora continua: La evaluación del éxito proporcionó datos y conocimiento que se utilizaron luego para mejorar los procesos y metodologías de gestión de proyectos. Esto contribuye a la mejora continua y a la optimización de futuras iniciativas.
    \item Rendición de cuentas: Permitió al product owner rendir cuentas a los sponsors y stakeholders y demostrar que gestionaron los recursos de manera adecuada y responsable, cumpliendo con los objetivos establecidos.
\end{enumerate}

A continuación, se detalla la metodología que se utilizó y los resultados obtenidos.

\section{Metodología de evaluación}

La metodología de evaluación se basó en tomar diferentes métricas y evaluarlas en función de las distintas fases de despliegue del proyecto.

Se evaluaron dichas métricas en un estado inicial (previo al despliegue del proyecto), como punto de partida, llamado \emph{baseline} y luego, se tomó una primer evaluación, a los 12 meses de haber iniciado (y desplegado el proyecto) y se re-evaluaron dichas métricas para poder ver la evolución.

Las métricas que se evaluaron son las detalladas inicilamnete en la Sec.\ref{success:metrics} y rondan en tres ejes principales:

\begin{enumerate}
    \item Métricas de experienca de usuario (\ref{success:metric_ux})
    \item Métricas de adopción de la herramienta (\ref{success:adoption}) 
    \item Métricas de ahorro y optimización (\ref{success:timesaving})
    \item Métricas de Auditoría y Cumplimiento (\ref{success:compliance})
\end{enumerate}

\section{Resultados cuantitativos}

A continuación, se detallan las mediciones para cada metrica comparando el punto de partida (línea base) y el resultado al cabo de 12 meses del primer despliegue a producción.

En primer lugar, se puede ver en la \emph{Tabla \ref{tab:tiempo_respuesta}}, que los tiempos de repuesta están en su mayoría alineados con lo planificado, aunque en el caso de la gestión de usuario, hasta ese momento no se había logrado alcanzar la meta propuesta.


\begin{table}[h!]
    \centering
    \begin{tabular}{lcccccc}
        \hline
        \textbf{ms por tx} & \textbf{Línea Base} & \textbf{12meses Prom} & \textbf{min} & \textbf{max} \\ \hline
        Gestion de Geografia \ref{usecases:geolevel1}& N/A & 750 & 750 & 1020\\ \hline
        Gestion de estructura 2 \ref{usecases:geolevel2}& N/A & 857 & 750 & 1020\\ \hline
        Gestion de usuarios \ref{usecases:useradmin}& N/A & 1123 & 750 & 1020\\ \hline
        Gestion de Acceso \ref{usecases:accessmgmnt} & N/A & 950 & 750 & 1020\\ \hline
        Navegacion \ref{usecases:browse}& N/A & 1050 & 750 & 1020\\ \hline
        Gestion Contenido \ref{usecases:contentaccess}& N/A & 600 & 750 & 1020\\ \hline
    \end{tabular}
    \caption{Tiempos de respuesta en ms por transacción}
    \label{tab:tiempo_respuesta}
\end{table}

Respecto de las métricas de adopción, se puede ver en la \emph{Tabla \ref{tab:adopcion_contenidos}}, que como el sistema es nuevo, se partió de una base de accesos, reportes, usuarios y países en 0 (cero), y al cabo de 12 meses se pudo evaluar el tráfico, en función de los contenidos y equipos que se fueron sumando. 

Respecto de los objetivos de cobertura en términos de reportes, el avance fue significativo, cubriendo en todos los casos, la mayoría de los reportes, usuarios y países existentes. Queda espacio para seguir expandiendo y es un punto que se abordará en el trabajo futuro propuesto, sobre el final de esta tesis.

\begin{table}[h!]
    \centering
    \begin{tabular}{lcccc}
        \hline
        \textbf{Métricas de Adopción} & \textbf{Línea Base} & \textbf{12meses} & \textbf{Objetivo} \\ \hline
        Accesos Mensuales & 0 & 7000(prom)/10000(max) & N/A \\ \hline
        Reportes & 0 & 307 & 550 \\ \hline
        Usuarios Únicos & 0 & 1785 & 2500 \\ \hline
        Países & 0 & 42 & 50 \\ \hline
    \end{tabular}
    \caption{Métricas de Adopción y Contenidos}
    \label{tab:adopcion_contenidos}
\end{table}

Finalmente, en la \emph{Tabla \ref{tab:ahorros_optimizacion}}, se hizo una evaluación de los tickets de soporte creados de modo automático y los pedidos de acceso resueltos. Si historicamente, la creación manual de un ticket de soporte, llevaba alrededor de 5 minutos, es fácil ver que solamente en tiempo de ingreso de datos, se ahorraron el equivalente a unas 86 horas de trabajo consolidadas, que aunque en período de 12 meses, podría ser un tiempo marginal, representa menos desperdicio y una mejor experiencia para los usuarios, con tareas que ya no exsiten de modo manual.

Respecto a los pedidos de acceso que se registraron, por cada pedido de acceso, el tiempo de resolución orignal, pasando por múltiples personas y equipos hasta que era finalmente rechazado o aprobado, se calculaba en unas 48hs hábiles (en promedio), con lo cual, en este caso, los ahorros de tiempo por la simplificación del proceso de gestión de accesos, representó un 95\% de ahorro por pedido, lo cual se traduce en casi 2750 días de espera eliminados.


\begin{table}[h!]
    \centering
    \begin{tabular}{lcccc}
        \hline
        \textbf{Métricas de Ahorro y Optimizacion} & \textbf{Línea Base} & \textbf{12meses} & \textbf{Objetivo} \\ \hline
        Tickets de Soporte Creados Automáticamente & 0 & 1036 & N/A \\ \hline
        Pedidos de Acceso & 0 & 1447 & N/A \\ \hline
    \end{tabular}
    \caption{Métricas de Ahorro y Optimizacion}
    \label{tab:ahorros_optimizacion}
\end{table}



\section{Resultados cualitativos}

Adicionalmente, se hizo una medición cualitativa del universo de usuarios, identificando aquellos de mayor peso en la organización y luego todo el resto. Esto es de fundamental relevancia, porque cuanto más alto en la escala de la organización se llega, mayor exposición se logra de la solución, y si bien esto puede representar un riesgo desde el punto de vista de posibles fallas de funcionamiento, ha sido muy bien recibido y ha generado aliados y promotores de la plataforma, con peso significativo en la organización. El detalle de los perfiles de los usuarios, se detalla a continuación, en la en la \emph{Tabla \ref{tab:perfiles_alto_impacto}}.

\begin{table}[h!]
    \centering
    \begin{tabular}{lccc}
        \hline
        \textbf{Perfiles de Usuarios de Alto Impacto} & \textbf{Línea Base} & \textbf{12meses} \\ \hline
        Vice Presidentes Sr & 0 & 2 \\ \hline
        Vice Presidentes & 0 & 2 \\ \hline
        Vice Presidentes Asociados & 0 & 2 \\ \hline
        Directores Ejecutivos & 0 & 5 \\ \hline
        Directores & 0 & 15 \\ \hline
        Otros & 0 & 1759 \\ \hline
    \end{tabular}
    \caption{Perfiles de Usuarios de Alto Impacto}
    \label{tab:perfiles_alto_impacto}
\end{table}

\section{Impacto en seguridad y cumplimiento}

Un objetivo clave del proyecto era abordar desafíos de cumplimiento que la organización enfrentaba debido a la falta de registros centralizados y mecanismos de auditoría eficientes. Para ello, el sistema de logs de auditoría automáticos y completos que permitió una captura detallada y precisa de todas las actividades relevantes.

El proyecto abordó estos desafíos mediante la implementación de un sistema de logs de auditoría automatizados, que incluía las siguientes características:
\begin{enumerate}
\item \emph{Captura Automática de Datos}: Todas las actividades relevantes se registraban automáticamente{,} eliminando la necesidad de intervención manual y garantizando la integridad de los datos.
\item \emph{Registros Completos y Detallados}: Los logs incluían información detallada sobre cada acción{,} incluyendo el usuario responsable{,} la fecha y hora{,} y la naturaleza de la acción.
\item \emph{Acceso y Monitoreo en Tiempo Real}: Los responsables de cumplimiento podían acceder a los registros en tiempo real{,} permitiendo una supervisión continua y proactiva.
\end{enumerate}

Beneficios Obtenidos:
La implementación del sistema de logs de auditoría completos y automatizados resultó en varios beneficios para la organización:
\begin{itemize}
    \item \emph{Mejora en el Cumplimiento}: La capacidad de mantener registros detallados y accesibles permitió a la organización cumplir con las regulaciones de manera más efectiva y reducir el riesgo de sanciones.
    \item \emph{Aumento de la Transparencia}: La disponibilidad de logs detallados mejoró la transparencia dentro de la organización, facilitando la rendición de cuentas y la toma de decisiones basada en datos.
    \item \emph{Eficiencia Operativa}: La automatización de la captura de datos y la reducción de auditorías manuales permitieron una asignación más eficiente de recursos y una mejora en la eficiencia operativa.
    \item \emph{Detección Temprana de Incidentes}: El acceso en tiempo real a los registros permitió la detección y respuesta rápida a incidentes, mejorando la seguridad y la gestión de riesgos.
\end{itemize}

El proyecto fue exitoso en resolver los desafíos de cumplimiento mediante la implementación de un sistema de logs de auditoría completos y automatizados. La organización no solo mejoró su capacidad de cumplir con las regulaciones, sino que también aumentó la transparencia, eficiencia operativa y capacidad de respuesta ante incidentes. Este enfoque proactivo y basado en datos ha fortalecido la posición de la organización en términos de cumplimiento y gestión de riesgos.

\section{Conclusiones de la evaluación}

El proyecto, luego de 12 meses, logró resultados destacados e impactos tangibles. siendo ahora nuevo estándard dentro de la corporación y permitiendo un nivel de simplicidad, acceso y eficiencia que parecia imposible. Una solución simple (aunque con cierta complejidad técnica) logró atender los desafíos propuestos en incluso ser adoptada globalmente.

