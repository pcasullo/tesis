% \chapter{Conclusiones}

% En esta sección, el objetivo es destacar el aporte de esta tesis como así también del proyecto en cuestión.

% \section{Resumen de los logros principales}

% Durante el desarrollo de esta tesis, se han alcanzado varios logros significativos en el proyecto ReportHub:

% \begin{enumerate}
    
%     \item Identificación de Problemas y Limitaciones: Se identificaron las principales limitaciones y problemas enfrentados por los usuarios y los equipos de datos en la organización, tales como la gestión de accesos y el manejo del portafolio de soluciones de inteligencia de negocios.
    
%     \item Desarrollo de la Propuesta: Se diseñó y desarrolló ReportHub, una aplicación web con una arquitectura orientada a servicios (SOA) y principios de diseño minimalistas y defensivos. Esta herramienta es escalable, auditable y centrada en las necesidades de la organización.
    
%     \item Implementación y Despliegue: Se implementó un enfoque metodológico robusto para el desarrollo y despliegue de ReportHub, incluyendo una estrategia de despliegue en fases y un plan de comunicación y capacitación eficaz.
    
%     \item Evaluación de Resultados: La evaluación del proyecto mostró resultados cuantitativos y cualitativos positivos que resaltaron la mejora en la gestión de contenidos y accesos, así como un impacto significativo en la seguridad y el cumplimiento.
% \end{enumerate}

% \section{Relevancia del proyecto para la organización}
% El proyecto ReportHub ha demostrado ser de gran relevancia para la organización al mejorar la eficiencia en la gestión de datos y contenidos. Esta herramienta ha proporcionado una solución integral a los problemas identificados, lo que ha resultado en una mejor toma de decisiones y una mayor agilidad en los procesos internos.

% \section{Impacto estratégico a largo plazo}
% A largo plazo, ReportHub tiene el potencial de transformar la forma en que la organización gestiona sus datos y contenidos. La herramienta contribuirá a una mayor transparencia y control sobre los datos, facilitando el cumplimiento de regulaciones y mejorando la seguridad. Además, su diseño escalable permite una fácil adaptación a futuras necesidades y expansiones.

% \section{Próximos pasos sugeridos (roadmap evolutivo)}
% Para continuar con el éxito de ReportHub, se sugieren los siguientes pasos:
% \begin{itemize}
%     \item Expansión de Funcionalidades**: Incorporar nuevas funcionalidades basadas en el feedback de los usuarios para mejorar aún más la experiencia y la utilidad de la herramienta.
%     \item Monitoreo y Mejora Continua**: Establecer un sistema de monitoreo continuo para evaluar el desempeño de ReportHub y realizar mejoras periódicas basadas en los resultados obtenidos.
%     \item Capacitación Continua**: Ofrecer capacitaciones regulares a los usuarios para asegurar un uso óptimo de la herramienta y maximizar su impacto positivo en la organización.
% \end{itemize}
    

% \newpage

\chapter{Conclusiones de esta tesis}

\section{Resumen de los logros principales}

Esta tesis, resume, el recorrido completo para implementar una solución, a un problema que era recurrente y no estaba resuelto de ningún modo razonable. Desde la conceptualización del problema y su formalización, hasta el desarrollo, implementación y despliegue productivo, midiéndo el impacto en el entorno donde se desplegó. Los logros que documenta la tesis, sobre este proyecto, están descriptos a continuación:

\begin{itemize}
    \item Identificación de Problemas y Limitaciones: 
    
El primer logro significativo fue la identificación de las principales limitaciones y problemas enfrentados por los usuarios y los equipos de datos en la organización. Este paso inicial fue fundamental para definir el alcance y los objetivos del proyecto. Mediante un exhaustivo análisis de la situación inicial, se detectaron problemas críticos como la gestión de accesos, el consumo de contenidos, y el manejo del portafolio de soluciones de inteligencia de negocios. Estas limitaciones representaban obstáculos importantes para la eficiencia operativa y la toma de decisiones en la organización. La identificación de estos problemas permitió establecer una base sólida para el desarrollo de soluciones efectivas.

    \item Desarrollo de la Propuesta:

El siguiente logro fue el diseño y desarrollo de ReportHub, una aplicación web innovadora (para el entorno corporativo en cuestión) con una arquitectura orientada a servicios (SOA) y principios de diseño minimalistas y defensivos. ReportHub se concibió como una herramienta escalable y auditable, centrada en satisfacer las necesidades específicas de la organización. La aplicación incluyó funcionalidades clave como la gestión de usuarios, la administración de contenidos, y la navegación de catálogos, entre otras. El desarrollo de ReportHub implicó un enfoque colaborativo, involucrando a diversas partes interesadas para asegurar que la solución final fuera robusta y alineada con los objetivos estratégicos de la organización.

    \item Implementación y Despliegue:

La implementación de ReportHub se llevó a cabo mediante un enfoque metodológico robusto que incluyó una estrategia de despliegue en fases. La primera fase, conocida como el Producto Mínimo Viable (MVP), se realizó con un grupo piloto para probar y refinar la herramienta antes de su expansión a más áreas y usuarios. 

Esta estrategia permitió identificar y solucionar problemas tempranos, asegurando una transición suave hacia la fase de expansión. Además, se desarrolló un plan de comunicación y capacitación eficaz para garantizar que todos los usuarios estuvieran bien informados y capacitados en el uso de ReportHub. Esta fase también incluyó la creación de mecanismos de soporte post-lanzamiento para atender cualquier necesidad o problema que pudiera surgir después de la implementación.

    \item Evaluación de Resultados:

Finalmente, la evaluación del proyecto mostró resultados cuantitativos y cualitativos positivos. Los datos recopilados durante la evaluación indicaron mejoras significativas en la gestión de contenidos y accesos, así como un impacto notable simplificando y automatizando la seguridad y el cumplimiento de normativas. 

Los usuarios reportaron una mayor eficiencia en sus tareas diarias y una mejor experiencia general con la nueva herramienta. Estos resultados no solo validaron las hipótesis iniciales del proyecto, sino que también destacaron la efectividad de ReportHub como una solución integral para los desafíos identificados.

\end{itemize}
En resumen, los logros alcanzados durante el desarrollo de esta tesis han sido numerosos y significativos. La identificación de problemas y limitaciones, el desarrollo de una propuesta innovadora, la implementación y despliegue efectivos, y la evaluación positiva de resultados, han contribuido a mejorar la eficiencia operativa y la toma de decisiones en la organización. Estos logros subrayan la importancia del proyecto ReportHub y su impacto positivo en la organización.

\section{Relevancia del proyecto para la organización}

El proyecto ReportHub ha demostrado ser de gran relevancia para la organización por varias razones. En primer lugar, ha mejorado significativamente la eficiencia en la gestión de datos y contenidos. Antes de la implementación de ReportHub, los usuarios enfrentaban múltiples desafíos relacionados con la gestión de accesos y el manejo de contenidos. Estos problemas no solo ralentizaban los procesos internos, sino que también afectaban la calidad de las decisiones tomadas. Con la introducción de ReportHub, estos desafíos se han abordado de manera efectiva, permitiendo una gestión más fluida y eficiente de los datos.


La importancia de ReportHub también se refleja en su capacidad para proporcionar una solución integral a los problemas identificados. La herramienta ha sido diseñada para ser escalable y adaptable, lo que significa que puede evolucionar con las necesidades cambiantes de la organización. Esta flexibilidad es crucial en un entorno empresarial dinámico, donde las demandas y los requisitos pueden cambiar rápidamente. ReportHub ha demostrado ser capaz de adaptarse a estas demandas, asegurando que la organización siempre tenga acceso a una herramienta eficaz y actualizada.


Además, ReportHub ha mejorado la transparencia y el control sobre los datos. La capacidad de auditar y monitorear las actividades dentro de la plataforma ha permitido a la organización mantener un control más estricto sobre el uso y la gestión de los contenidos/reportes. Esto no solo ha mejorado la seguridad y el cumplimiento de normativas, sino que también ha aumentado la confianza de los usuarios en la herramienta. Saber que sus actividades están siendo monitoreadas de manera efectiva ha llevado a una mayor adherencia a las políticas y procedimientos establecidos.


Otro punto a destacar, es que la experiencia del usuario ha sido mejorada de modo consitente y de inicio a fin, de manera completa.ReportHub ha permitido a los usuarios acceder a la información que necesitan de manera rápida y eficiente. 


En términos de impacto estratégico, ReportHub ha establecido una base sólida para futuras iniciativas de gestión de datos. La implementación exitosa de esta herramienta ha demostrado que la organización está comprometida con la mejora continua y la innovación y a su vez cuenta con herramientas que permiten una gestión, basada en datos, del portafolio de productos de BI y futuras inversiones.


En resumen, la relevancia del proyecto ReportHub para la organización es innegable. Ha mejorado la eficiencia operativa, la transparencia y el control sobre los reportes/contenidos de BI. Estos beneficios han tenido un impacto positivo en la organización, posicionándola de caras a demandas crecientes de gestión efectiva y eficiente de sus datos y procesos.

\section{Impacto estratégico a largo plazo}

A largo plazo, ReportHub tiene el potencial de transformar la forma en que la organización gestiona sus datos y contenidos. Esta transformación no solo se limita a la mejora de procesos internos, sino que también tiene implicaciones estratégicas de gran alcance. A continuación se detallan los aspectos clave del impacto estratégico de ReportHub a largo plazo:

\begin{itemize}
    \item Transparencia y Control:

Una de las principales ventajas estratégicas de ReportHub es su capacidad para proporcionar una mayor transparencia y control sobre los datos. En un entorno regulado, donde el cumplimiento de normativas es crucial, tener una herramienta que permita auditar y monitorear las actividades es esencial. ReportHub facilita el cumplimiento de regulaciones al proporcionar un registro claro y accesible de todas las actividades. Esta transparencia no solo ayuda a evitar sanciones y penalidades, sino que también mejora la reputación de la organización ante reguladores y stakeholders externos.

    \item Adaptabilidad y Escalabilidad:

El diseño escalable y adaptable de ReportHub permite que la herramienta evolucione con las necesidades de la organización. A medida que la organización crece y se enfrenta a nuevos desafíos, ReportHub puede adaptarse para seguir siendo una herramienta útil y relevante. Esta capacidad de adaptación es crucial para mantener la competitividad en un entorno empresarial en constante cambio. Además, la escalabilidad de ReportHub significa que puede ser implementada en diferentes áreas y departamentos de la organización, ampliando su impacto y beneficios.

    \item Mejora Continua:

La capacidad de ReportHub para facilitar la mejora continua es otro aspecto clave de su impacto estratégico. La herramienta permite un monitoreo constante del rendimiento y el uso, lo que facilita la identificación de áreas de mejora. Esta capacidad de monitoreo y ajuste continuo asegura que la herramienta siempre esté optimizada para ofrecer el mejor rendimiento posible. Además, la recopilación de feedback de los usuarios permite realizar mejoras basadas en las necesidades reales y cambiantes de los usuarios, asegurando que ReportHub siga siendo una herramienta valiosa a largo plazo.

    \item Seguridad y Cumplimiento:

La seguridad de los datos es una preocupación creciente en el entorno empresarial actual. ReportHub ha sido diseñada con un enfoque robusto en la seguridad, incluyendo características como la seguridad a nivel de fila (Row Level Security, RLS) y la capacidad de auditar y monitorear todas las actividades. Estas características aseguran que los datos estén protegidos contra accesos no autorizados y que cualquier actividad inusual sea detectada y abordada de manera oportuna. La capacidad de cumplir con las normativas de seguridad y protección de datos es una ventaja estratégica importante, ya que ayuda a evitar sanciones y a mantener la confianza de los stakeholders.

\end{itemize}

En resumen, el impacto estratégico de ReportHub a largo plazo es significativo. La herramienta proporciona transparencia y control, es adaptable y escalable, facilita la mejora continua, asegura la protección de datos y cumplimiento normativo, y posiciona a la organización como innovadora y competitiva. Estos beneficios estratégicos aseguran que ReportHub seguirá siendo una herramienta valiosa y relevante para la organización en el futuro.

\section{Próximos pasos sugeridos (roadmap evolutivo)}

Para asegurar el éxito continuo de ReportHub y maximizar su impacto positivo en la organización, se sugieren los siguientes pasos como parte de un roadmap evolutivo:

\begin{itemize}
    \item Expansión de Funcionalidades:
    
Uno de los primeros pasos sugeridos es la expansión de las funcionalidades de ReportHub. Basándose en el feedback de los usuarios, se pueden identificar áreas donde la herramienta puede mejorar aún más su utilidad. Por ejemplo, se podrían incorporar funcionalidades avanzadas de análisis de datos, integración con otras herramientas de gestión empresarial, y mejoras en la interfaz de usuario para facilitar una experiencia más intuitiva. La expansión de funcionalidades no solo mejorará la experiencia del usuario, sino que también aumentará el valor de la herramienta para la organización.

    \item Monitoreo y Mejora Continua: 

Establecer un sistema de monitoreo continuo es crucial para asegurar que ReportHub siga funcionando de manera óptima. Este sistema debe incluir la recopilación de datos sobre el rendimiento de la herramienta, así como el feedback de los usuarios. Con esta información, se pueden realizar ajustes y mejoras periódicas para mantener la herramienta actualizada y relevante. Además, el monitoreo continuo permite identificar de manera proactiva cualquier problema o área de mejora, asegurando que la herramienta siempre esté optimizada para ofrecer el mejor rendimiento posible.

    \item Capacitación Continua:

La capacitación continua de los usuarios es otro paso importante en el roadmap evolutivo de ReportHub. Ofrecer capacitaciones regulares asegura que los usuarios estén siempre informados sobre las nuevas funcionalidades y mejoras de la herramienta. Además, la capacitación ayuda a maximizar el impacto positivo de la herramienta al asegurar que los usuarios sepan cómo utilizarla de manera efectiva. Se pueden organizar talleres, seminarios web y sesiones de formación en línea para mantener a los usuarios actualizados y comprometidos.

    \item Expansión al resto de los equipos de BI:
    
De las métricas presentadas, se deduce que todavía queda trabajo por hacer para lograr una cobertura completa de los contenidos y reportes existentes en el mundo de BI de la compañía. En función de los beneficios logrados hasta el momento, se evalúa continuar hasta lograr una cobertura total.

\end{itemize}


En resumen, los próximos pasos sugeridos para el roadmap evolutivo de ReportHub incluyen la expansión de funcionalidades, el monitoreo y mejora continua, y la expansión a nuevos equipos de BI, para lograr una cobertura completa de la gestión de las soluciones existentes. Estos pasos asegurarán que ReportHub siga siendo una herramienta valiosa y relevante para la organización, maximizando su impacto positivo a largo plazo.

