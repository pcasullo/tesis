%\begin{center}
%\large \bf \runtitulo
%\end{center}
%\vspace{1cm}
\chapter*{\runtitulo}

\noindent 

% En las grandes corporaciones multinacionales coexisten múltiples tecnologías y prácticas de \gls{bi} que abarcan todo el espectro del \gls{techstack}, desde el \gls{dw}, pasando por los procesos de \gls{etl} \cite{inmon1992} hasta herramientas de visualización y análisis. A pesar de los intentos por establecer estándares tecnológicos y de \Gls{gob} \cite{isaca2012}, la descentralización de iniciativas entre áreas funcionales y niveles geográficos genera un ecosistema complejo, heterogéneo y difícil de armonizar. Este escenario plantea importantes desafíos tanto en la integración de soluciones como en la medición de su impacto en el negocio, donde la relación entre adopción e valor resulta poco evidente\cite{davenport2006}. En este contexto, definir \glspl{kpi} de adopción se vuelve crítico para gestionar de manera eficiente el portafolio de soluciones de \gls{bi}, aunque su implementación requiere superar barreras metodológicas y organizacionales. Asimismo, la \gls{ux} se ve afectada por la falta de consistencia en el manejo de los accesos y servicios, lo que refuerza la necesidad de establecer enfoques más integrados y estandarizados en la gestión de estas herramientas.

% Estas corporaciones, suelen tener prácticas de definiciones de estándares tecnológicos, que definen herramientas o tecnologías oficiales a ciertos productos y a proveedores, con el fin de simplificar y lograr sinergías tecnológicas entre las herramientas seleccionadas y reducir las combinaciones, según distintos casos de uso\cite{jacobson1986}.

% \bigskip

% \noindent\textbf{Palabras claves:} \emph{\gls{bi}, \gls{dw}, \Gls{techstack}, \Gls{ux},  \Gls{pm}, \Gls{auto}, \Gls{dam}, \Gls{gproc}, \Gls{gob}.}

Las organizaciones multinacionales operan con ecosistemas de \gls{bi} altamente heterogéneos, en los que conviven múltiples tecnologías a lo largo del \gls{techstack}, desde \gls{dw} y procesos de \gls{etl} \cite{inmon1992} hasta diversas herramientas analíticas. A pesar de los esfuerzos por establecer estándares y marcos de \Gls{gob} \cite{isaca2012}, la descentralización funcional y geográfica dificulta la armonización de prácticas y limita la capacidad de medir el impacto real de la las iniciativas de analytics, especialmente cuando la relación entre adopción y valor no es evidente \cite{davenport2006}.

Esta tesis aborda este vacío documentando el proceso de desarrollo, desde la concepción y el diseño, hasta la implementación de una plataforma integral orientada a unificar la gestión de reportes corporativos, estandarizar la experiencia de acceso y habilitar la medición consistente de \glspl{kpi} de adopción. La solución propone un modelo de gobernanza que centraliza metadatos, consolida criterios de clasificación y optimiza y formaliza flujos de solicitud y aprobación. Asimismo, incorpora principios de \gls{ux} y gestión de portafolio para mejorar la trazabilidad, reducir duplicación y facilitar el descubrimiento de activos analíticos. La arquitectura incluye mecanismos que permiten integrar reglas de seguridad existentes y fortalecer procesos de control organizacional. Los resultados obtenidos muestran mejoras en visibilidad, coherencia operativa y capacidad de análisis sobre el uso efectivo de las soluciones de \gls{bi}, contribuyendo a la optimización del ecosistema analítico corporativo.
\bigskip

Palabras clave: \emph{\gls{bi}, \gls{dw}, \Gls{techstack}, \Gls{ux}, \Gls{pm}, \Gls{auto}, \Gls{dam}, \Gls{gproc}, \Gls{gob}.}