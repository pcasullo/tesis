\makeglossaries
\newglossaryentry{LDAP}{
    name=LDAP,
    description= LDAP significa Protocolo Ligero de Acceso a Directorios (Lightweight Directory Access Protocol) y es un protocolo de software que se utiliza para buscar acceder y administrar servicios de directorio a través de una red como bases de datos de información de usuarios cuentas y contraseñas para la autenticación centralizada de aplicaciones y servicios en una organización.,
    plural=LDAP
}

\newglossaryentry{dw_descr}{
    name=data warehouse,
    description= base de datos masiva que se utiliza para realizar análisis de información de negocio para la toma de decisiones,
    plural=Data Warehouses
}

\newglossaryentry{gob}{
    name=gobierno,
    description= descripcion de prueba,
    plural=LDAP
}

\newglossaryentry{af}{
    name=área funcional,
    description=areas de la organización encargadas de actividades no vinculadas a la industria en particular sino también consideradas de soporte de negocio ejemplos pueden ser \gls{rh},
    plural=áreas funcionales.
}

\newglossaryentry{ti}{
    name=IT,
    description=Information Technology,
    plural=IT
}

\newglossaryentry{pm}{
    name=portfolio management,
    description=Metodologías y procesos para poder administrar una cartera de activos,
    plural=portfolio management
}

\newglossaryentry{auto}{
    name=automatización,
    description=Reemplazo de trabajo manual por mecanismos automáticos que eviten la intervención humana.,
    plural=automatizaciones
}

\newglossaryentry{dam}{
    name=decentralized Access Management,
    description=Implementación de procesos sistemas y roles que permite eque la gestión de acceso se realice de forma descentralizada.
    plural=decentralized Access Management
}

\newglossaryentry{gproc}{
    name=global process,
    description=proceso estandard y unificado seguido de por todas las subsidiarias de una compañía global para lograr un objetivo determinado.,
    plural=global processes
}

\newglossaryentry{lux}{
    name=lean UX,
    description = User Experience minimalista,
    plural=lean UXs
}

\newglossaryentry{wapp}{
    name=web application,
    description=Aplicación con arquitectura diseñada para correr en entornos web y ser accedida mediante browsers de internet.,
    plural=web applications
}

\newglossaryentry{techstack}{
    name=stack tecnológico,
    description=conjunto de tecnologías utilizadas para implementar una solución desde el software de base tecnologías de base de datos sistema operativo lenguaje de programación librerías.
    plural=stacks tecnológicos
}

\newacronym{ldap}{LDAP}{Lightweight Directory Access Protocol}
\newacronym{etl}{ETL}{Extract, Transform and Load}
\newacronym{dw}{DW}{Data Warehouse}
\newacronym{rh}{RR.HH.}{Recursos Humanos}
\newacronym{it}{TI}{Tecnología de la información}
\newacronym{bi}{BI}{Business Intelligence}
\newacronym{kpi}{KPI}{Key Performance Indicator}
\newacronym{ux}{UX}{User Experience}
\newacronym{soa}{SOA}{Service Oriented Architecture}
\newacronym{uat}{UAT}{User Acceptance Test}
\newacronym{dev}{DEV}{Development / Desarrollo}
\newacronym{prd}{PRD}{Production/Producción}
\newacronym{mvc}{MVC}{Model View Controller}



