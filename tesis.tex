\documentclass[11pt,a4paper,twoside]{tesis}
% SI NO PENSAS IMPRIMIRLO EN FORMATO LIBRO PODES USAR
%\documentclass[11pt,a4paper]{tesis}

\usepackage{graphicx}
\usepackage[utf8]{inputenc}
%\usepackage[english]{datatool}
\usepackage[spanish]{babel}
\usepackage{csquotes}
\usepackage{pifont}
\usepackage[left=3cm,right=3cm,bottom=3.5cm,top=3.5cm]{geometry}
\usepackage[backend=biber,style=numeric]{biblatex} % estilo APA, podés cambiarlo
\addbibresource{tesis.bib} % nombre de tu archivo .bib

% Paquete para glosario
\usepackage[toc, acronym]{glossaries}
\usepackage{booktabs}
\usepackage{xcolor}

\usepackage{xurl}

\usepackage{tablefootnote}


% Paquetes para gráficos
\usepackage{tikz}
\usetikzlibrary{automata, arrows.meta, positioning, shapes.multipart, shapes.geometric, shapes, shadows, backgrounds,fit,calc}
%\usetikzlibrary{positioning,shapes,arrows.meta,shadows,backgrounds}

\tikzset{
  block/.style={rectangle, draw, fill=blue!10, rounded corners,
                minimum height=1cm, minimum width=2.8cm, align=center},
  block2/.style={rectangle, draw, fill=red!10, rounded corners,
                minimum height=1cm, minimum width=2.8cm, align=center},
  block3/.style={rectangle, draw, fill=yellow!10, rounded corners,
                minimum height=1cm, minimum width=2.8cm, align=center},
  db/.style={cylinder, draw, fill=green!10, aspect=0.25, shape border rotate=90,
             minimum height=1cm, minimum width=2.5cm, align=center},
  db2/.style={cylinder, draw, fill=red!10, aspect=0.25, shape border rotate=90,
             minimum height=1cm, minimum width=2.5cm, align=center},
%ellipse/.style={ellipse, draw, fill=blue!10, align=center},
  arrow/.style={-{Stealth}, thick},
%  every node/.style={font=\sffamily},
  comp/.style={draw, rounded corners, minimum width=2.8cm, minimum height=1cm, align=center, fill=violet!20, drop shadow},
  webserver/.style={draw, rounded corners, minimum width=2.8cm, minimum height=1cm, align=center, fill=yellow!20, drop shadow},
  appserver/.style={draw, rounded corners, minimum width=2.8cm, minimum height=1cm, align=center, fill=blue!20, drop shadow},
  lb/.style={draw, ellipse, minimum width=2.5cm, minimum height=1.2cm, align=center, fill=green!20, drop shadow},
  server/.style={draw, densely dashed, rounded corners, inner sep=4mm},%, fill=gray!5}
  server2/.style={draw, densely dashed, rounded corners, inner sep=4mm, color=green},%, fill=blackgray!5}
  sharedFS/.style={cylinder, draw, fill=red!10, aspect=0.25, shape border rotate=90,
             minimum height=1cm, minimum width=2.5cm, align=center}, 
}


% Paquete para comentarios en el pdf.
\usepackage{xargs} 
\usepackage[colorinlistoftodos,prependcaption,textsize=tiny]{todonotes}
\newcommandx{\unsure}[2][1=]{\todo[linecolor=red,backgroundcolor=red!25,bordercolor=red,#1]{#2}}
\newcommandx{\change}[2][1=]{\todo[linecolor=blue,backgroundcolor=blue!25,bordercolor=blue,#1]{#2}}

%habilito el glosario
\makeglossaries
\newglossaryentry{LDAP}{
    name=LDAP,
    description= LDAP significa Protocolo Ligero de Acceso a Directorios (Lightweight Directory Access Protocol) y es un protocolo de software que se utiliza para buscar acceder y administrar servicios de directorio a través de una red como bases de datos de información de usuarios cuentas y contraseñas para la autenticación centralizada de aplicaciones y servicios en una organización.,
    plural=LDAP
}

\newglossaryentry{dw_descr}{
    name=data warehouse,
    description= base de datos masiva que se utiliza para realizar análisis de información de negocio para la toma de decisiones,
    plural=Data Warehouses
}

\newglossaryentry{gob}{
    name=gobierno,
    description= descripcion de prueba,
    plural=LDAP
}

\newglossaryentry{af}{
    name=área funcional,
    description=areas de la organización encargadas de actividades no vinculadas a la industria en particular sino también consideradas de soporte de negocio ejemplos pueden ser \gls{rh},
    plural=áreas funcionales.
}

\newglossaryentry{ti}{
    name=IT,
    description=Information Technology,
    plural=IT
}

\newglossaryentry{pm}{
    name=portfolio management,
    description=Metodologías y procesos para poder administrar una cartera de activos,
    plural=portfolio management
}

\newglossaryentry{auto}{
    name=automatización,
    description=Reemplazo de trabajo manual por mecanismos automáticos que eviten la intervención humana.,
    plural=automatizaciones
}

\newglossaryentry{dam}{
    name=decentralized Access Management,
    description=Implementación de procesos sistemas y roles que permite eque la gestión de acceso se realice de forma descentralizada.
    plural=decentralized Access Management
}

\newglossaryentry{gproc}{
    name=global process,
    description=proceso estandard y unificado seguido de por todas las subsidiarias de una compañía global para lograr un objetivo determinado.,
    plural=global processes
}

\newglossaryentry{lux}{
    name=lean UX,
    description = User Experience minimalista,
    plural=lean UXs
}

\newglossaryentry{wapp}{
    name=web application,
    description=Aplicación con arquitectura diseñada para correr en entornos web y ser accedida mediante browsers de internet.,
    plural=web applications
}

\newglossaryentry{techstack}{
    name=stack tecnológico,
    description=conjunto de tecnologías utilizadas para implementar una solución desde el software de base tecnologías de base de datos sistema operativo lenguaje de programación librerías.
    plural=stacks tecnológicos
}

\newacronym{ldap}{LDAP}{Lightweight Directory Access Protocol}
\newacronym{etl}{ETL}{Extract, Transform and Load}
\newacronym{dw}{DW}{Data Warehouse}
\newacronym{rh}{RR.HH.}{Recursos Humanos}
\newacronym{it}{TI}{Tecnología de la información}
\newacronym{bi}{BI}{Business Intelligence}
\newacronym{kpi}{KPI}{Key Performance Indicator}
\newacronym{ux}{UX}{User Experience}
\newacronym{soa}{SOA}{Service Oriented Architecture}
\newacronym{uat}{UAT}{User Acceptance Test}
\newacronym{dev}{DEV}{Development / Desarrollo}
\newacronym{prd}{PRD}{Production/Producción}
\newacronym{mvc}{MVC}{Model View Controller}





\begin{document}

%%%% CARATULA

\def\autor{Pablo S. Casullo}
\def\tituloTesis{Experience Report: \vspace{.2cm} \\ ReportsHub}
\def\runtitulo{Experience Report: ReportsHub}
\def\runtitle{Experience Report: ReportsHub}
\def\director{Diego Garbervetsky}
%\def\codirector{Master Yoda}
\def\lugar{Buenos Aires, 2025}
\input{caratula}

%%%% ABSTRACTS, AGRADECIMIENTOS Y DEDICATORIA
\frontmatter
\pagestyle{empty}
%\begin{center}
%\large \bf \runtitulo
%\end{center}
%\vspace{1cm}
\chapter*{\runtitulo}

\noindent 

En las grandes corporaciones multinacionales coexisten múltiples tecnologías y prácticas de \gls{bi} que abarcan todo el espectro del \gls{techstack}, desde el \gls{dw}, pasando por los procesos de \gls{etl} \cite{inmon1992} hasta herramientas de visualización y análisis. A pesar de los intentos por establecer estándares tecnológicos y de \Gls{gob} \cite{isaca2012}, la descentralización de iniciativas entre áreas funcionales y niveles geográficos genera un ecosistema complejo, heterogéneo y difícil de armonizar. Este escenario plantea importantes desafíos tanto en la integración de soluciones como en la medición de su impacto en el negocio, donde la relación entre adopción e valor resulta poco evidente\cite{davenport2006}. En este contexto, definir \glspl{kpi} de adopción se vuelve crítico para gestionar de manera eficiente el portafolio de soluciones de \gls{bi}, aunque su implementación requiere superar barreras metodológicas y organizacionales. Asimismo, la \gls{ux} se ve afectada por la falta de consistencia en el manejo de los accesos y servicios, lo que refuerza la necesidad de establecer enfoques más integrados y estandarizados en la gestión de estas herramientas.

Estas corporaciones, suelen tener prácticas de definiciones de estándares tecnológicos, que definen herramientas o tecnologías oficiales a ciertos productos y a proveedores, con el fin de simplificar y lograr sinergías tecnológicas entre las herramientas seleccionadas y reducir las combinaciones, según distintos casos de uso\cite{jacobson1986}.

\bigskip

\noindent\textbf{Palabras claves:} \emph{\gls{bi}, \gls{dw}, \Gls{techstack}, \Gls{ux},  \Gls{pm}, \Gls{auto}, \Gls{dam}, \Gls{gproc}, \Gls{gob}.}




\cleardoublepage
%%\begin{center}
%\large \bf \runtitle
%\end{center}
%\vspace{1cm}
\chapter*{\runtitle}

\noindent In large multinational corporations, multiple business intelligence technologies and practices coexist, spanning the entire technology stack, from extraction, transformation, and loading (ETL) processes \cite{inmon1992} to visualization and analytics tools. Despite attempts to establish technological and governance standards \cite{isaca2012}, the decentralization of initiatives across functional areas and geographic levels creates a complex, heterogeneous ecosystem that is difficult to harmonize. This scenario poses significant challenges both in integrating solutions and in measuring their business impact, where the relationship between adoption and impact is often unclear \cite{davenport2006}. In this context, defining adoption metrics becomes critical to coherently managing the business intelligence solutions portfolio, although its implementation requires overcoming methodological and organizational barriers. Likewise, the user experience is affected by the lack of consistency in access and services, reinforcing the need to establish more integrated and standardized approaches in the management of these tools.
These corporations typically have practices for defining technological standards, which establish official tools or technologies for certain products and vendors, with the aim of simplifying and achieving technological synergies among selected tools and reducing combinations according to different use cases \cite{jacobson1986}.

\bigskip

\noindent\textbf{Keywords:} Data Warehouse, Business Intelligence, Portfolio Management, Automation, Distributed Access management, Global Process. % OPCIONAL: comentar si no se quiere

\cleardoublepage
%\chapter*{Agradecimientos}

\noindent En primer lugar, quiero agradecer a mi país, \textbf{La República Argentina}, por haberme dado educación pública de calidad, en todos los niveles escenciales de mi formación académica. 

\noindent A \textbf{Diego}, \textit{mi director}, por su generosa guía e incansable energía para acompañarme en este proceso. 

\noindent A \textbf{Eduardo}, \textit{mi padre}, que desde su pasión por la innovación y la tecnología me acercó al mundo de las computadoras, cuando todavía mi edad se contaba con un sólo dígito. 

\noindent A \textbf{Stella}, \textit{mi madre}, que siempre me apoyó incondicionalmente en lo que hizo falta, contra vientos y mareas. 

\noindent A \textbf{Facu}, \textit{primer Doctor en la familia y entrañable hermano}, con quien compartimos viajes, pasiones, desafíos y esperanzas. 

\noindent A \textbf{Carla}, \textit{mi mano derecha} en esta aventura y ser un motor inagotable que impulsó el desarrollo de este proyecto, con un equipo de gente espectacular y contra viento y marea. 

\noindent A \textbf{Franco}, \textit{colega y compañero} de este proceso, por su revisión y feedback valioso, para hacer de esta tesis, algo mejor.

\noindent A \textbf{Esteban}, \textit{un enorme amigo} que me sumó la vida, con quien he compartido charlas y momentos de notable singularidad, y que con su lectura "naif" de este texto, hizo aportes fundamentales para cuidar la claridad y el impacto de este trabajo. 

\noindent Y finalmente, y en el lugar más especial, a \textbf{\emph{Dante, Joaco y Sil}}, que me soportaron durante este proceso, con paciencia y comprensión. % OPCIONAL: comentar si no se quiere

\cleardoublepage
\hfill \textit{A Dante y Joaco, desde mi corazón, todo mi amor.}
  % OPCIONAL: comentar si no se quiere

\cleardoublepage
\tableofcontents

\mainmatter
\pagestyle{headings}

%%%% ACA VA EL CONTENIDO DE LA TESIS

\chapter{Motivacion}

\section{Contexto organizacional}
%{\begin{small}%
%\begin{flushright}%
%\it
%There's nothing for me now.
%I want to learn the ways of\\ the Force and become a Jedi like my father. \\
%--Luke Skywalker
%\end{flushright}%
%\end{small}%
%\vspace{.5cm}}

En las grandes corporaciones multinacionales, existen múltiples tecnologías para resolver distintos desafíos de análisis e inteligencia de negocios, a lo largo de todo el stack tecnológico que va desde los sistemas \gls{etl} \cite{inmon1992}, distintos repositorios de almacenamiento para datos estructurados y no estructurados, y herramientas de exploración y explotación de datos que se asocian a la capa de presentación. En esta última coexisten planillas de cálculo (xls) y software específico de visualización para crear reportes y tableros de información, tales como Microstrategy, Cognos, PowerBI, Tableau o Qlik, entre otros.

Estas corporaciones suelen definir estándares tecnológicos que determinan herramientas o proveedores oficiales, con el fin de simplificar y lograr sinergias entre las tecnologías seleccionadas y reducir las combinaciones según distintos casos de uso \cite{jacobson1986}.

Adicionalmente, es común que las iniciativas de inteligencia de negocios estén descentralizadas y distribuidas tanto en áreas funcionales (marketing, ventas, compras, finanzas), como en distintos niveles geográficos, que abarcan desde equipos corporativos hasta instancias globales, regionales y locales.

\section{Situación inicial y limitaciones detectadas}

Este contexto de distribución de recursos en áreas y geografías, combinado con la multiplicidad de herramientas “aprobadas” por dichas corporaciones, genera una combinación de componentes para la creación de distintas soluciones que, aun siguiendo las mejores prácticas de gobernanza tecnológica y de arquitectura \cite{isaca2012}, devienen en un ecosistema complejo, desarmonizado y lleno de particularidades.

Poder establecer el valor real e impacto que estas soluciones tienen en el negocio es un desafío, ya que en muchos casos la relación entre adopción e impacto es indirecta y no trivial \cite{davenport2006}. A su vez, la experiencia de los distintos usuarios varía significativamente: solicitar acceso o encontrar los enlaces de cada herramienta resulta inconsistente, incoherente y dependiente de qué equipo haya desarrollado la solución.

\section{Problemas principales para usuarios y equipos de datos}
\subsection{Gestión de accesos}

La gestión de accesos junto con la aplicación de políticas de seguridad una vez otorgados los mismos, variaban absolutamente y quedaban definidas por el criterio de cada equipo que desarrollaba los contenidos, sin haber coherencia alguna.
Algunos ejemplos de variantes incluyen:

Enviar un email a quien dio las capacitaciones (en caso que las haya habido). Muchas veces la persona encargada de la instrucción o entrenamiento de usuarios era también la encargada de controlar el acceso y otorgarlo. En otros casos, actuaba como intermediario para poder llegar a quien era responsable de otorgar los accesos adecuados.
Solicitar acceso mediante herramientas de tickets a equipos encargados de la operación de dichos tableros/reportes. En equipos de proyecto de tamaños medio o grandes (con un número de integrantes excediendo la decena), es común que haya gente dedicada exclusivamente a tareas operativas, cuyo objetivo es por un lado garantizar la continuidad y máxima disponibilidad de las soluciones como así también en ocasiones se encargan de ejecutar las actividades que habilitan acceso y perfiles de seguridad adecuados a los usuarios.
Solicitar acceso a alguien conocido si puede hacer de nexo para dar con el contacto indicado.

Finalmente, por el diseño (o falta de diseño) de estos procesos, hay múltiples intermediarios en dichos pedidos y otorgamientos, cuando en realidad, en escencia, deberían existir idealmente dos roles:

\begin {itemize}
\item quien solicita acceso.
\item quien aprueba, define y otorga (en una sola persona). 
\end {itemize}

Por estas complejidades accidentales, muchas veces, quien lo aprueba y define, no es quien se encarga de que se haga el otorgamiento efectivo. Y es aquí donde aparece una de las mayores oportunidades de simplificación y opimización.

\subsection{Consumo de contenidos}

Para poder consumir o utilizar reportes es fundamental saber de su existencia, conocer su ubicación (normalmente son enlaces dentro de redes internas) y tener acceso a los mismos.
Algunos ejemplos acerca de cómo acceder, pueden incluir:

\begin{enumerate}
    \item Usuarios reciben enlaces de acceso por emails o en documentos de entrenamientos y luego (con suerte) los almacenan como atajos en sus navegadores (con los problemas de tener hardcoded links que luego con el tiempo pueden variar, apuntar a versiones obsoletas de dichos reportes o incluso quedar deprecados conforme algunos reportes son decomisionados).
    \item Páginas de intranet donde se publican los puntos de acceso (muchas veces implementadas como portales de acceso, provistos por las mismas herramientas de visualización).
    \item Usuarios reciben reportes como archivos adjuntos en correos electrónicos.
\end{enumerate}

\subsection{Manejo del portafolio de soluciones de inteligencia de negocios}

Para poder hacer un manejo efectivo de un portafolio de soluciones y de las tecnologías que se utilizan, es necesario poder entender:

¿Qué soluciones hay disponibles y en qué tecnologías están desarrolladas (inventario)?

¿Qué áreas de negocio cubren y quienes son responsables?

¿Qué nivel de adopción tienen?

¿Qué usuarios tienen acceso y no deberían?

¿Qué usuarios necesitan acceso y no tienen?

¿Qué usuarios tienen acceso y no lo usan?

¿Qué usuarios tienen alto nivel de adopción?

¿Qué volumen de pedidos de acceso manejan, según las audiencias objetivo?

¿Hay duplicidad de contenidos, hechos por áreas afines pero sin colaborar?

¿Hay patrones de uso de reportes que tengan correlación con el desempeño de áreas de negocio?

La respuesta a cada una y en particular a todas estas preguntas, en el entorno descripto, es de un esfuerzo que no permite tener información en tiempo y forma ni de modo adecuado, repetible y consistente de manera constante. Poder responder cada pregunta en un contexto tan heterogéneo, implica un nivel de trabajo manual, armonización y alineación de criterios y definiciones que simplemente lo vuelven imposible, en la escala de estas organizaciones.

\newpage
En resumen, los impactos negativos descriptos anteriormente pueden sintetizarse en la siguiente lista: 

\begin{itemize}
\item Falta de un repositorio central y común donde buscar/encontrar y solicitar acceso a los contenidos necesarios, genera dificultades para poder dar con los contenidos.
\item Procesos inconsistentes y altamente variables.
\item Falta de transparencia en cuanto a puntos de contactos (personas responsables) de reportes.
\item Métodos de aprobación de múltiples pasos y manuales, que dependen de horarios laborales, feriados distribuidos en varias husos horarios y múltiples geografías.
\item Separación entre nivel de aprobación (hecha por la persona responsable) de un acceso y el nivel de ejecución de dicho acceso (hecha por operadores, una vez que la persona responsable ha definido qué acceso corresponde).
\item Falta de información necesaria para el aprobador, acerca de rol, país, función y otros elementos que ayudan a determinar si un acceso debe o no ser otorgado a un usuario.
\item La ausencia de un repositorio centralizado agrava los problemas de consistencia y coherencia. Esto se traduce en dificultades para asegurar estándares de seguridad homogéneos, en la duplicación de esfuerzos.
\item El estado fragmentado limita la capacidad de contar con indicadores confiables y oportunos para la toma de decisiones de portafolio. 
\item La falta de métricas unificadas de adopción impide evaluar el impacto de las soluciones desarrolladas, dificultando la gestión adecuada del portafolio de inteligencia de negocios.
\end{itemize}


\section{Aporte de la Tesis}
Esta \textit{tesis}, en formato de Experience Report, tiene como objetivo, profundizar cómo fue el proceso de desarrollo de una solución que atendió a los desafíos mencionados, detallando desde su concepción en el contexto organizacional, hasta su despliegue y medición del impacto, dentro de una organización de estas características.

Los contenidos a desarrollar incluyen:

\textbf{Definiciones Preliminares:} Conjunto de definiciones básicas que nos introducen en el dominio del problema.

\textbf{Propuesta:} Descripción detallada conceptual de la solución, en función de sus requerimientos funcionales y no funcionales. \cite{zave1979} \cite{yeh1980}

\textbf{Arquitectura:} Descripción de los componentes y sus relaciones como también algunos patrones de diseño aplicados. \cite{yourdon1979} \cite{erl2005} \cite{fowler2003}

\textbf{Desarrollo y Despliegue:} Explicación de la metodología aplicada y entregas iterativas. 

\textbf{Evaluación:} Monitoreo y evaluación de las métricas clave y factores de éxito.

\textbf{Lecciones aprendidas:} Hallazgos y aprendizajes de la experiencia en el proyecto.

\textbf{Conclusiones:} Síntesis final del trabajo, con implicancias finales.

%\unsure { es una prueba de unsure para ver si funciona, está piola}

\chapter{Definiciones Preliminares}

Las definiciones preliminares son aquellas que nos permiten profundizar en los elementos básicos del dominio de nuestro contexto. A continuación se presentan las que aplican a conceptos, roles y elementos clave utilizados a lo largo de este trabajo.

\section{Definiciones de industria y corporaciones multinacionales}
\begin{description}
    \item [KPIs:] Key Performance Indicators (Indicadores clave de desempeño).
\end{description}

\section{Definiciones de inteligencia de negocios}

\begin{description}
    \item [Inteligencia de negocios:] (Business intelligence)
    \item [Data Warehouse:]
\end{description}

\section{Definiciones de diseño y arquitectura de software}
\begin{description}
    \item[Stack tecnológico:] 
    \item [Web App:]
    \item [RLS:]
    \item [LDAP:]
    \item [API:]
    \item [SOA:]
    \item [Cohesión:]
    \item [Acoplamiento:]
    \item [Web Server:]
    \item [Load Balancer:]
    \item [Web Application Server:]
    \item [Back end:]
    \item [Client Server:]
    \item [Caché:]
    \item [Tolerancia a fallas:]
    \item [NFS:]
    \item [Database Server:]
    \item [MVC:] Model View Controller \cite{pope1988}. Patrón de diseño mediante el cual, se separan en tres capas bien definidas, para mayor flexibilidad, mantenibilidad y escalabilidad.
    
\end{description}

\section{Definiciones de metodologías}
    
\begin{description}

\item[Waterfall/Cascada:] Descripción de la Metodología waterfall.
\item[Agile/Agil:] Descripción de la Metodología ágil.
\item[DevOps:] Metodología DevOps.
\item[Unit tests:] Pruebas unitarias.
\item[Integration tests:] Pruebas de integración.
\item[Dev:] Desarrollo. Entorno para desarrollo y pruebas de unidad y de integración.
\item[UAT:] User Acceptance Test. Entorno de pruebas de usuario y validaciones finales antes de hacer pasajes a producción.
\item[Pr:] Production/Producción. Entorno de operaciones productivo.

\end{description}

\section{Definiciones propias del proyecto}

\begin{description}

\item[ReportHub:] Plataforma propuesta para consolidar y armonizar la publicación de reportes y tableros, gestión de accesos y monitoreo de métricas de uso de BI.

\item[Administradores:] Usuarios con privilegios para gestionar contenidos, accesos y usuarios dentro de la plataforma. Se distinguen tres tipos:
    \begin{itemize}
        \item \textbf{Admin Global:} Gestiona unidades geográficas globales y áreas de información a nivel global.
        \item \textbf{Admin Local:} Gestiona áreas de información dentro de su región geográfica asignada.
    \end{itemize}

\item[Creadores de Contenidos:] Usuarios responsables de generar y publicar contenidos dentro de la plataforma.

\item[Consumidores de Contenidos:] Usuarios que acceden y utilizan los contenidos publicados en el portal, pudiendo también marcar favoritos y realizar solicitudes de acceso.

\item[Aprobadores:] Usuarios que evalúan y aprueban solicitudes de acceso a contenidos según la configuración de seguridad definida.

\item[Estructura de contenidos:] Jerarquía de organización de contenidos en dos niveles:
    \begin{enumerate}
        \item \textbf{Nivel 1:} Unidades geográficas (Global, Europa, América, Asia, África y países correspondientes).
        \item \textbf{Nivel 2:} Áreas de información (temáticas específicas, únicas dentro de cada región).
    \end{enumerate}

\item[Metadatos de Contenido:] Información asociada a cada contenido publicado, incluyendo:
    \begin{itemize}
        \item Título
        \item Descripción
        \item Imagen miniatura
        \item Tipo de contenido (archivo o URL)
        \item Segmentos de usuarios y objetivos de frecuencia de uso
        \item Aprobadores
        \item Configuración de provisión de acceso
    \end{itemize}

\end{description}

\section{Definiciones de diseño y arquitectura de software}

\begin{description}

    \item [Stack tecnológico:] Conjunto de tecnologías y herramientas utilizadas para construir y ejecutar una aplicación o sistema \gls{techstack}.
    \item [Web App:] Aplicación que se ejecuta en un navegador web y accede a servicios a través de Internet \gls{wapp}.
    \item [RLS:] \gls{rls}, directrices o políticas que definen cómo se maneja y protege la información sensible.
    \item [LDAP:] \gls{ldap}, un protocolo de red para acceder a servicios de directorio de información.
    \item [API:] \gls{api}, un conjunto de rutinas y protocolos de programación que permiten la creación de software.
    \item [SOA:] \gls{soa}, un enfoque de diseño en el que tanto las aplicaciones como los componentes de las aplicaciones son services intercambiables.
    \item [Cohesión:] La medida en que las funciones relacionadas se agrupan en un módulo o clase \cite{yourdon1979}.
    \item [Acoplamiento:] La dependencia entre módulos o componentes de software \cite{yourdon1979}.
    \item [\Gls{webserver}:] Servidor que sirve páginas web y gestiona las solicitudes HTTP.
    \item [\Gls{loadbalancer}:] Dispositivo de red que distribuye la carga de trabajo entre múltiples servidores.
    \item [Web Application Server:] Servidor que gestiona aplicaciones web y proporciona servicios adicionales como escalado y seguridad.
    \item [Back end:] Parte del sistema que maneja la lógica de negocio, el almacenamiento de datos y la interacción con el front end \gls{backend}.
    \item [\Gls{clientserver}:] Modelo de arquitectura de red en el que los clientes solicitan servicios al servidor.
    \item [\Gls{cache}:] Componente de hardware o software que almacena copias temporales de datos para acceder a ellos más rápidamente.
    \item [Tolerancia a fallas:] La capacidad de un sistema para continuar funcionando a pesar de errores o fallos.
    \item [NFS:] \Gls{nfs}, sistema de archivos central y compartido que permite a múltiples computadoras acceder a archivos de forma simultánea.
    \item [\Gls{dbserver}:] Un servidor que gestiona bases de datos y permite el acceso y la manipulación de datos.
    \item [MVC:] Model View Controller \cite{pope1988}. Patrón de diseño mediante el cual, se separan en tres capas bien definidas, para mayor flexibilidad, mantenibilidad y escalabilidad.
    
\end{description}

\section{Definiciones de metodologías}
    
\begin{description}

\item[Waterfall/Cascada:] Una metodología de desarrollo de software en la que el proceso se divide en fases secuenciales y se asume que cada fase se completa antes de pasar a la siguiente \cite{royce1970}.
\item[Agile/Agil:] Una metodología de desarrollo de software que promueve la iteración y la colaboración, y que se adapta a los cambios a medida que el proyecto avanza.% \cite{agile2001}.
\item[DevOps:] El término "DevOps" fue acuñado por Patrick Debois en 2009 para nombrar su conferencia "DevOpsDays", la cual se inspiró en la charla "10 despliegues por día" de John Allspaw y Paul Hammond en la Conferencia Velocity de 2009. El término se creó al combinar Development (desarrollo) y Ops (operaciones) para resolver la brecha entre los dos equipos.\cite{allspaw2009}.
\item[Unit tests:] Pruebas unitarias.
\item[Integration tests:] Pruebas de integración.
\item[Dev:] Desarrollo. Entorno para desarrollo y pruebas de unidad y de integración.
\item[UAT:] User Acceptance Test. Entorno de pruebas de usuario y validaciones finales antes de hacer pasajes a producción.
\item[Pr:] Production/Producción. Entorno de operaciones productivo.

\end{description}

\section{Definiciones propias del proyecto}

\begin{description}

\item[ReportHub:] Plataforma propuesta para consolidar y armonizar la publicación de reportes y tableros, gestión de accesos y monitoreo de métricas de uso de BI.

\item[Administradores:] Usuarios con privilegios para gestionar contenidos, accesos y usuarios dentro de la plataforma. Se distinguen tres tipos:
    \begin{itemize}
        \item \textbf{Admin Global:} Gestiona unidades geográficas globales y áreas de información a nivel global.
        \item \textbf{Admin Local:} Gestiona áreas de información dentro de su región geográfica asignada.
    \end{itemize}

\item[Creadores de Contenidos:] Usuarios responsables de generar y publicar contenidos dentro de la plataforma.

\item[Consumidores de Contenidos:] Usuarios que acceden y utilizan los contenidos publicados en el portal, pudiendo también marcar favoritos y realizar solicitudes de acceso.

\item[Aprobadores:] Usuarios que evalúan y aprueban solicitudes de acceso a contenidos según la configuración de seguridad definida.

\item[Estructura de contenidos:] Jerarquía de organización de contenidos en dos niveles:
    \begin{enumerate}
        \item \textbf{Nivel 1:} Unidades geográficas (Global, Europa, América, Asia, África y países correspondientes).
        \item \textbf{Nivel 2:} Áreas de información (temáticas específicas, únicas dentro de cada región).
    \end{enumerate}

\item[Metadatos de Contenido:] Información asociada a cada contenido publicado, incluyendo:
    \begin{itemize}
        \item Título
        \item Descripción
        \item Imagen miniatura
        \item Tipo de contenido (archivo o URL)
        \item Segmentos de usuarios y objetivos de frecuencia de uso
        \item Aprobadores
        \item Configuración de provisión de acceso
    \end{itemize}

\end{description}


\chapter{Propuesta}
\section{Visión y objetivos del proyecto}

Por todo lo anteriormente descrito, es que surge este proyecto, impulsando un espacio de consolidación y armonización de publicación, gestión de accesos y monitoreo de métricas de uso para la adecuada gestión del portafolio de BI. Asimismo, unificar el acceso y ofrecer una experiencia homogénea y consistente, independiente de la tecnología de implementación, se presenta como una oportunidad para mejorar tanto la eficiencia de los equipos de datos como la satisfacción de los usuarios finales.

De la complejidad de estas problemáticas surgió la necesidad de lograr una solución que permita:

Concentrar en un solo lugar la oferta de reportes/tableros ofrecida por los distintos equipos.
Permitir a los equipos que publican estos elementos, realizar una gestión de sus usuarios, incluyendo la asignación de roles si aplican restricciones de visibilidad de datos.
Establecer objetivos de adopción y medirlos de modo consistente, de manera totalmente independiente a la tecnología de implementación que se haya utilizado.
Darle a los potenciales usuarios una experiencia homogénea y consistente, de modo agnóstico a las tecnologías utilizadas mediante técnicas de ingeniería de software.
A los usuarios, almacenar atajos o accesos directos “resilientes” que resistan cambios de URLs de reportes y a la vez conserven atributos de seguridad para que compartiendo los enlaces la seguridad se mantenga.

\section{Principios de diseño de ReportHub}

El diseño de la solución se basó en un conjunto de principios arquitectónicos que aseguraron no solo la cobertura de las necesidades funcionales del momento, sino también la capacidad de evolucionar en el tiempo. Estos principios tuvieron como propósito garantizar que la plataforma fuera escalable, permitiendo crecer en volumen de usuarios, contenidos y procesos sin comprometer el rendimiento. A su vez, aseguraron un mantenimiento eficiente, reduciendo la complejidad técnica y posibilitando la incorporación de nuevas funcionalidades de forma ágil y con bajo costo operativo.
Otro eje fundamental fue la facilidad de uso, que permitió que cada rol interactuara con la solución de manera simple, intuitiva y orientada a sus tareas principales, evitando barreras de adopción. Asimismo, se estableció un marco que permitió cumplir con las normas internas de auditoría, control de calidad y gobierno de la información, garantizando la trazabilidad y responsabilidad en cada acción ejecutada dentro del sistema.
Finalmente, los principios incorporaron las mejores prácticas en seguridad y estándares internacionales, asegurando la protección de datos, la correcta gestión de accesos y la interoperabilidad con diferentes tecnologías, lo que fortaleció la resiliencia y confiabilidad de la solución en contextos de negocio dinámicos y regulados.

La solución debe cumplir con los siguientes principios de diseño:

\subsection{\Gls{wapp}}
\label{principios:webapp}
Será una aplicación web, que contará con elementos de presentación en el browser, una capa de lógica de negocio en un servidor y se apoyará en una base de datos relacional para almacenar la información y meta información necesaria para la configuración, uso y auditoría de actividades.

\subsection{\Gls{lux} minimalista y defensiva}
\label{principios:leanUx}
Experiencia de usuario simple e intuitiva para cada uno de los roles en que los usuarios operen la solución, ya que un mismo usuario, puede tener contenidos para publicar, ser administrador del sistema y eventualmente consumidor de contenidos publicados por otros usuarios:
 \begin{itemize}
 \item \textit{Administradores, roles:} Admin Global / Admin Local.
 \item \textit{Creadores de Contenidos, rol:} Creador.
 \item \textit{Consumidores de contenidos, rol:} Consumidor.
 \end{itemize}

La interfaz, debe evitar que el usuario cargue información inválida mediante reglas de negocio intuitivas y ya incorporadas a la navegación y cuando esto nos sea posible, validará la información que el usuario ingrese y proveerá feedback inmediato en el momento de las interacciones para poder corregir problemas.

\subsection{Tener una \Gls{soa}}
\label{principios:soa}
Ser agnóstico de las tecnologías en las que se han producido los contenidos que se publicarán minimizando el acoplamiento técnico y a la vez funcionando con una alta cohesión. Este punto permite garantizar la interoperabilidad con diferentes tecnologías y a la vez el soporte de procesos comunes, sin estar condicionados por las tecnologías de implementación.

\subsection{Descentrado y Escalable}
\label{principios:federado}
Permitir la descentralización de la gestión de contenidos, accesos y configuraciones de modo de poder asignar en distintos niveles y equipos organizacionales las facultades para el auto servicio de sus contenidos. Esto evita los cuellos de botella de estructuras centralizadas y fomenta que se tomen las decisiones adecuadas en cada lugar adecuado de la organización, siguiendo procesos consistentes garantizados por el sistema.

\subsection{Auditable}
\label{princpipios:auditable}
Cada acción debe ser auditable, para poder cumplir con normas internas de auditorías de calidad de procesos y responsabilidad, en particular a la hora de administrar usuarios y accesos.

\section{Casos de uso principales}
\label{usecases:main}
Los principales casos de uso, tienen como objetivo detallar en un nivel conceptual, las funcionalidades básicas y escenarios que iba a soportar el sistema. 

\subsection{Creación de estructuras geográficas para organizar los contenidos (Admin Global).}
\label{usecases:geolevel1}
Los contenidos del sistema debían estar organizados en una jerarquía de dos niveles. El primer nivel, compuesto de unidades geográficas, tenía como objetivo agrupar por geografía y a su vez, poder otorgar niveles de administración descentralizados a distintos Admins de geografías para administrar áreas de información. Se utilizan como valores válidos, los correspondientes a las distintas unidades geográficas:
	-Global
	-Europa
	-America
	-Asia
	-Africa
	y luego países
   	Todos estarán agrupados en un mismo nivel.

%\subsection{Creación de áreas de información para organizar los contenidos(Admin Global/Local).}
\label{usecases:geolevel2}
El segundo nivel de organización eran áreas de información y permitían una agrupación lógica de los contenidos en función de las temáticas que cubrían, como por ejemplo finanzas, marketing, ventas, etc. No existía un listado predefinido de qué áreas de información existíann a nivel local y podían ser creadas libremente por los administradores a nivel geográfico. La única restricción: los nombres de las áreas en una región geográfica debían ser únicos.


Por ejemplo:

\begin{itemize}
\item Global
    \begin{itemize}
        \item Marketing
        \item Cumplimiento
        \item Ventas
        \item Finanzas
        \item Recursos Humanos, etc.
    \end{itemize}
	\item America
		\begin{itemize}
		    \item Marketing
            \item Legales, etc.
        \end{itemize}
	\item Argentina
        \begin{itemize}
            \item Fuerza de Ventas
            \item Finanzas
            \item Oportunidades, etc.	
        \end{itemize}
\end{itemize}

\subsection{Gestión de Usuarios.}
\label{usecases:useradmin}

La creación y administración de usuarios, consistió en poder registrar usuarios que ya eran parte del directorio corporativo como usuarios de este sistema y asignarles roles de Admins globales, locales, de áreas de información y/o eventualmente como consumidores. Del directorio corporativo debían importarse los siguientes datos: handle de usuario de la compañía, nombre completo, dirección de correo electrónico.

	Es importante aclarar que un Admin tenía la capacidad de crear secciones, contenidos y manejar usuarios, pero no necesariamente acceso a los mismos contenidos que publicó, ya que son aspectos guiados por diferentes criterios de seguridad y las personas responsables de los contenidos eran quienes debían aprobar y decidir quienes debían tener acceso.
		La forma en la que los permisos iban a ser asignados de acuerdo a los roles se detalla en la tabla \ref{tab:funcxrol}.

\begin{table} [h]
    \centering
    \begin{tabular}{|c|c|c|c|c|}\hline
        \textbf{Función/rol}&  \textbf{Admin global}&  \textbf{Admin local}&  \textbf{Creador}& \textbf{Aprobador}\\\hline
         Crear áreas globales&  \ding{51}&  &  & \\\hline
         Crear áreas locales& \ding{51}&  \ding{51}&  & \\\hline
         Crear contenidos&  \ding{51}&  \ding{51}&  \ding{51}& \\\hline
         Administrar usuarios&  \ding{51}&  \ding{51}&  \ding{51}& \ding{51}\\\hline
         Asignar Roles $^1$&  \ding{51}&  \ding{51}&  \ding{51}& \ding{51}\\\hline
         Aprobar accesos&  &  &  \ding{51}& \ding{51}\\\hline
    \end{tabular}
    \caption{Funciones por rol (\ding{51})}
    \label{tab:funcxrol}
\end{table}


        
        Todos los usuarios iban a poder asignar a otros usuarios, según las reglas de la tabla \ref{tab:asignaciones}:


\begin{table} [h]

    \centering
    \begin{tabular}{|c|c|c|c|c|}\hline
         \textbf{Rol / puede asignar}&  \textbf{Admin global}&  \textbf{Admin local}&  \textbf{Creador}& \textbf{Aprobador}\\\hline
         Admin global&  \ding{51} &  \ding{51}&  \ding{51}& \ding{51}\\\hline
         Admin local&  &  \ding{51}&  \ding{51}& \ding{51}\\\hline
         Creador&  &  &  \ding{51}& \ding{51}\\\hline
         Aprobador&  &  &  & \ding{51}\\ \hline
    \end{tabular}
    \caption{Asignaciones de roles válidas (\ding{51})}
    \label{tab:asignaciones}
\end{table}

Estos mecanismos designados, también aplicaron a la hora de modificar los privilegios de un usuario. En caso de que un usuario hubiese sido desafectado del sistema, debía haber siempre un usuario alternativo como administrador global/local o creador de área de información o contenido. Si en algún momento algún usuario dejaba de pertenecer a la compañía, dejando a algún área sin administradores, entonces el sistema iba a asignar automáticamente a cargo de los elementos huérfanos del portal a quienes eran los responsables inmediatos superiores dentro del mismo, siguiendo la lógica de Aprobador - Creador - Admin Local y Admin Global en última instancia. Para Admins globales debía haber siempre al menos 2 e iba a ser parte de la configuración inicial del sistema.

\subsection{Gestión de contenidos (Creador/Admin).}
\label{usecases:contentadmin}
La publicación de contenidos se iba a hacer de modo que cada pieza de contenido estaría representada por los siguientes metadatos obligatorios:

\begin{itemize}
\item Título: Un texto denomina al contenido.
\item Descripción: Texto que da una breve descripción de lo que el contenido ilustra o representa.
\item Imagen miniatura: una imagen que representa el contenido, pudiendo ser un logo, un pequeño screenshot o cualquier elemento visual que permita reconocer al contenido publicado.
\item Tipo de Contenido: Los contenidos pueden ser de dos tipos, o bien archivos cargables o bien URLs a elementos que residan dentro de la intranet.
\item Segmentos de usuarios: Los segmentos de usuarios representan distintos perfiles de usuarios que ilustran la frecuencia esperada de uso de los contenidos de acuerdo al rol que cumplen en la organización. Puede ser que haya usuarios más operativos que necesitan acceder a un contenido particular con una frecuencia alta (por ejemplo varias veces por semana), y otros que tengan roles más tácticos o estratégicos y que utilicen estos contenidos con frecuencias menores (1 vez al mes o incluso de modo trimestral). Debe haber al menos 1 segmento de usuario (segmento por defecto llamado usuarios generales) y se le deben asignar objetivos de frecuencia de uso, en función de una cantidad de veces por unidades de tiempo. 
    Cantidad: Un número natural mayor a cero.
	Frecuencia: deberá ser algún valor de esta lista: “diario, semanal, mensual, trimestral, semestral, anual”

	De este modo, se iban a poder establecer objetivos de adopción según perfiles de usuario, y de acuerdo a la expectativa de cómo éstos, utilicen los contenidos para desempeñarse en sus procesos de negocio.

\item Aprobadores: El listado de aprobadores y por defecto quiénes iban a poder aprobar accesos.

\item Selección del modo de provisión de acceso:
    \begin{itemize}
    \item En caso que el tipo de Contenido sea un archivo, se aplica la lógica por defecto de acceso directo.
	\item En caso de que el tipo de Contenido sea un enlace a un reporte en otra tecnología, se opta entre dos modelos:
    \begin{enumerate}
        \item Un modelo de control de acceso integrado a un sistema de tickets, donde se debe especificar dentro de ese sistema qué grupo es el encargado de recibir el ticket y se configuran parámetros con los que se creará el ticket, según sean requeridos:
	       \begin{itemize}
	       \item Nombre del reporte al que se pide acceso.
	       \item Nombre de quien solicita el acceso. 
		   \item Handle de usuario.
		   \item Dirección de email.
		   \item Puesto en la compañía.
		   \item País de localización de quien lo pide.
           \end{itemize}
	    \item Si el reporte tenía un modelo de acceso basado en listas de distribución del directorio corporativo, entonces debían especificarse dichas listas y quedando asociadas al contenido (y se podían asociar a segmentos de usuarios).
        \end{enumerate}
	\item Si el reporte tenía seguridad basada en roles se defieron dos mecanismos posibles:
        \begin{enumerate}
            \item Listas de distribución especificas del directorio corporativo. En este caso se definía qué listas del directorio corporativo estaban asociadas a este contenido, para que el aprobador pueda utilizarlas en el momento de la aprobación.
		    \item Esquema propio de cada contenido con perfiles/roles particulares. En este caso, debían especificarse los end points y credenciales para poder integrar la administración.
        \end{enumerate}
    \end{itemize}
\end{itemize}

\subsection{Navegación del catálogo y solicitud de acceso (Consumidor).}
\label{usecases:browse}

El portal iba a tener dos modos de operación. Un primer modo predeterminado, que permitía utilizar los contenidos asignados, disponibles y ordenados según las unidades geográficas y áreas de información.
	También, una sección de “Favoritos” que iba a contener aquellos contenidos elegidos por el usuario como favoritos, que normalmente simplificaban el acceso a los de uso frecuente.
	Estaba la posibilidad de buscar contenidos por textos relacionados a los metadatos definidos en el caso de uso \ref{usecases:contentadmin}.

	El segundo modo, “catálogo”, donde el usuario consumidor iba a tener la posibilidad de ver los contenidos publicados, agrupados por unidades geográficas y áreas de información disponibles pero a los que no tiene acceso. Cada elemento iba a ser agregado a un “carrito de compras”, y una vez terminada la selección de los elementos para solicitar acceso, se enviaba un pedido formal de acceso. También estába la posibilidad de agregar un texto como solicitud, explicando para cada elemento (o de forma colectiva) por qué la persona necesitaba acceso para los elementos. Cuando el usuario completaba la acción de pedir acceso a los elementos del carrito de compras, se iban a disparar los procesos de aprobación, otorgamiento de accesos y notificación según se hayan definido, en cada elemento.


\subsection{Gestión de accesos y permisos (Creador/Admin).}
\label{usecases:accessmgmnt}

En el circuito de aprobaciones, quienes son aprobadores iban a recibir una notificación por email y también tendrían un ícono en el portal que les indicaría sus “tareas pendientes”. Tanto el email como el ícono, llevarían a los aprobadores a la interfaz donde evaluaríanlos pedidos en función de las solicitudes que recibieron para probarlos o rechazarlos de manera general (en lote) o particular (cada uno de ellos). En caso de rechazo de la solicitud, deb´pian colocar el motivo y en ambos (tanto positivo como negativo) casos el resultado del proceso iba a ser informado al usuario solicitante, tanto por email, como una notificación en el portal.
	Adicionalmente, si el contenido al que se solicitaba acceso, estaba cargado en el portal, el acceso se otorgaría de modo directo.
	En caso de que el contenido publicado tenga asociada seguridad manejada por listas de distribución, se haría la asociación en ese momento y si además tenía habilitada la configuración de administración de perfiles de aplicación, se debían asignar en el momento.
	Si por último, el contenido publicado, estaba implementado y operado bajo un modelo de soporte basado en tickets, con la información configurada en el momento de la creación del contenido en el portal, se generaría un ticket dirigido al equipo correspondiente, adjuntando la información necesaria para el alta y la aprobación del aprobador para que el equipo de mantenimiento cuente con el respaldo necesario para poder documentar y accionar el pedido, de acuerdo a las normas de cumplimiento de la empresa. En este caso, la notificación que recibiría el usuario que solicitó acceso tendría un texto que le informaría que su pedido fue aprobado y se se había creó un ticket nro XXX en su nombre, para que pudiera darle seguimiento con el equipo de operaciones.


\subsection{Navegacion y acceso a los contenidos.}
\label{usecases:contentaccess}

Las acciones de navegación dentro del portal tenían como objetivo poder recorrer los contenidos y accederlos. Para acceder un contenido, bastaría con clickear con el mouse sobre la el ícono que lo representa en el catálogo y esto desplegaría un nuevo Frame del navegador de internet con una dirección enmascarada al elemento en cuestión. Esto permitiría que el enlace se pueda guardar como acceso directo y a la vez compartir por el usuario con otras personas, sin exponer el link original y preservando el control de acceso primario. El efecto deseado era agregar un nivel de indirección de modo tal que al guardar el link como acceso directo, quienes publicaban los contenidos puedan modificar el link interno de acceso sin afectar los enlaces del portal.
	Finalmente, todas las acciones de búsqueda, navegación, gestión de usuarios y de accesos, serían registradas en una bitácora de eventos de modo tal de generar un historial de las transacciones ocurridas, para poder luego poder optimizar el funcionamiento del sitio, aplicar auditorías de cumplimento necesarias y a la vez obtener métricas que permitan medir los criterios de éxito del portal.	

\subsection{Logs de actividades y monitoreo de estadísticas.}
\label{usecases:logging}

El registro de actividades (logs) y el monitoreo de estadísticas de uso eran pilares fundamentales para garantizar su correcto funcionamiento, su transparencia y su evolución constante.

Los logs actuarían como un diario detallado de todo lo que ocurría dentro del sistema: desde eventos técnicos como errores, advertencias y cambios de configuración, hasta interacciones de los usuarios y transacciones críticas. 

Esta información era esencial para la auditabilidad, ya que permitiría reconstruir con precisión qué sucedió, cuándo y bajo qué condiciones. En entornos regulados o con altos estándares de seguridad, los logs son la evidencia que respalda el cumplimiento normativo, la trazabilidad de acciones y la detección de comportamientos anómalos o no autorizados.

Por otro lado, el monitoreo de estadísticas de uso recopilaría datos cuantitativos sobre el rendimiento y la utilización del sistema: tiempos de respuesta, consumo de recursos, patrones de tráfico, tasas de error, entre otros. Estos indicadores no solo permitirían evaluar el estado actual del sistema, sino que también habilitarían la observabilidad, es decir, la capacidad de comprender su comportamiento interno a partir de la información externa que genera.

La combinación de logs y métricas crearía un ecosistema de información que facilitaría la detección temprana de problemas y la resolución proactiva de incidentes. Por ejemplo, un aumento repentino en el tiempo de respuesta acompañado de errores registrados en los logs podría señalar un cuello de botella en la base de datos o una falla en un servicio externo.

Además, esta información era clave para instrumentar procesos de mejora continua. Analizando tendencias históricas y correlacionando eventos con métricas, es posible identificar áreas de optimización, priorizar desarrollos y medir el impacto real de los cambios implementados. Esto convierte al monitoreo y al registro en herramientas estratégicas, no solo operativas.

En resumen, sin logs ni monitoreo es como un barco navegando sin brújula ni bitácora: puede avanzar, pero carece de la capacidad de entender su rumbo, aprender de su experiencia y reaccionar ante imprevistos. Implementar un esquema robusto de registro y análisis de datos no solo garantiza auditabilidad y transparencia, sino que también potencia la eficiencia, la resiliencia y la capacidad de innovación de la plataforma.


\section{Alcance inicial y entregables previstos}
El alcance inicial comenzó con la implementación de los casos de uso para los roles de Administrador Global, Creador y Consumidor de contenidos. En una segunda fase se decidió incorporar al perfil de administrador Local para poder agregar una capa intermedia de administración que permita descentralizar y federar el gobierno de los contenidos de modo más eficiente.

Se priorizó también la publicación de contenido global en una primera instancia y luego en etapas sucesivas, luego de la implementación del rol local, se comenzó a expandir a otras unidades geográficas con el abordaje a los equipos locales de múltiples áreas.

\section{Definición de KPIs y métricas de éxito}
\label{success:metrics}

Para poder medir el avance y el éxito del proyecto se identificaron métricas de varios tipos:

\begin{enumerate}
    \item \textit{Experiencia de usuario}: \label{success:metric_ux}
	La primer métrica que se definió está asociada al tiempo de respuesta del portal durante las interacciones y en principio, lo que se buscó es que cada interacción, dure entre 300 y 500 milisegundos (sin incorporar el tiempo de latencia de red). Es decir que cada vez que el usuario disparaba una acción sobre el portal, el tiempo de respuesta combinado, desde que enviaba el estímulo hasta que recibía la respuesta completa (o el indicio de respuesta) no debía ser mayor a 1000 milisegundos. Cuando hablamos del indicio de respuesta, nos referimos a que a veces, por volumen de información que debe viajar desde el servidor de base de datos y web hasta el browser del usuario, simplemente no es posible, pero sí se puede comenzar a renderizar parcialmente o dar alguna indicación visual de que su solicitud fue hecha y que está en proceso de ser resuelta.

    \item \textit{Adopción}: \label{success:adoption}
    Para poder medir la adopción efectiva de la herramienta se estableció como métrica la cantidad de piezas de contenido creadas, cantidad de usuarios creadores, cantidad de usuarios asignados y finalmente actividad asociada según los tipos de transacciones definidos en los casos de uso.

    \item \textit{Ahorros de tiempo y recursos y minimización de errores}: \label{success:timesaving}
    
    Para poder hacer una medición de estos elementos se decidió hacer seguimiento de cantidad de tickets generados para gestión de accesos a grupos de soporte. Cantidad de accesos asignados mediante listas de distribución.

    \item \textit{Métricas de Auditoría y Cumplimiento}: \label{success:compliance}
    Para poder hacer una medición del cumplimento se decidió identificar qué nivel de cobertura de transacciones eran documentadas apuntando al 100 en tiempo real como objetivo.

\end{enumerate}
\chapter{Arquitectura}
\section{Principios de diseño}

La arquitectura que se buscó debía contar con los siguientes elementos: 

\begin{enumerate}
    \item Ser una Web-app (\ref{principios:webapp}) \label{arq:webapp}

El encarar el proyecto como una aplicación web, vs una arquitectura de cliente servidor tradicional con un cliente ejecutable e instalado en las máquinas clientes y el código del servidor corriendo en un server, permitió aprovechar las ventajas de mantenimiento y escalabilidad propias de las arquitecturas de las aplicaciones web. En escencia, siguen siendo cliente servidor, con la diferencia de que el código que se ejecuta en el cliente se descarga en el código web al que se accede desde el browswer de internet, de modo que al mantenimiento y el control de versiones, actualizaciones y mejoras se da de modo automático\footnote{En ciertos casos, es posible que haya código descargado en cachés locales, pero pueden invalidarse y actualizarse fácilmente como parte de las actualizaciones.} en cada máquina cliente.

Adicionalmente, con un diseño para dicha arquitectura, adecuado, se puede también escalar horizontalmente sin mayores esfuerzos y aumentar la tolerancia a fallas.

En el caso puntual de esta aplicación, se diseñaron dos configuraciones: una para entornos de desarrollo y pruebas y luego una para un entorno productivo, donde se tolera una carga de usuarios real y debe maximizarse la tolerancia a fallas, donde se agrega un tercer servidor web/de aplicaciones.

En los entornos de desarrollo y de test, siempre se prueba con la siguiente configuración:

\begin{figure}
    \centering
    \begin{tikzpicture} [node distance=1.8cm and 2.5cm]
 % Client
    \node[comp] (client) {Browser};
 % Load Balancer
    \node[lb, below=1cm of client] (f5) {LB};
    \node[server, fit=(f5)] (lbsrv) {};
    % Apache / Tomcat pairs
    \node[webserver, below left=1.2cm and 4cm of f5] (apache1) {Apache};
    \node[appserver, right=0.5cm of apache1] (tomcat1) {Tomcat};
    \node[server, fit=(apache1) (tomcat1)] (srv1) {};
    \node[webserver, below right=3.5cm and -4cm of f5] (apache2) {Apache};
%    \node[appserver, below=0.6cm of apache2] (tomcat2) {Tomcat};
    \node[appserver, right=0.5cm of apache2] (tomcat2) {Tomcat};
    \node[server, fit=(apache2) (tomcat2)] (srv2) {};
    \node[webserver, below right=1.2cm and 1cm of f5] (apache3) {Apache};
    \node[appserver, right=0.5cm of apache3] (tomcat3) {Tomcat};
    \node[server2, fit=(apache3) (tomcat3)] (srv3) {};

    %Shared FS
    \node[sharedFS, below =5cm of apache1] (shaFS) {NFS};    
    \node[server, fit=(shaFS)] (srvSharedFS) {};
    

    % Database
    \node[db, below=5cm of tomcat3] (db) {Oracle\\Database};
    \node[server, fit=(db)] (dbsrv) {};
     % Arrows
    \draw[-{Stealth[length=8pt]}] (client) -- (f5);
    \draw[-{Stealth[length=8pt]}] (f5.west) -| (srv1);
    \draw[-{Stealth[length=8pt]}] (f5.south) -- (srv2);
    \draw[-{Stealth[length=8pt]}] (f5.east) -| (srv3);
    \draw[-{Stealth[length=8pt]}] (apache1) -- (tomcat1);
    \draw[-{Stealth[length=8pt]}] (apache2) -- (tomcat2);
    \draw[-{Stealth[length=8pt]}] (apache3) -- (tomcat3);
    \draw[-{Stealth[length=8pt]}] (tomcat1) -- (apache1);
    \draw[-{Stealth[length=8pt]}] (tomcat2) -- (apache2);
    \draw[-{Stealth[length=8pt]}] (tomcat3) -- (apache3);
    \draw[-{Stealth[length=8pt]}] (tomcat1) |- (db.west);
    \draw[-{Stealth[length=8pt]}] (tomcat2.east) -| (db.north);
    \draw[-{Stealth[length=8pt]}] (tomcat3) -- (db.north);   
    \draw[-{Stealth[length=8pt,color=red]}, draw=red] (apache1.south) -| (shaFS.north);
    \draw[-{Stealth[length=8pt,color=red]}, draw=red] (apache2.south) |- (shaFS.east);
    \draw[-{Stealth[length=8pt,color=red]}, draw=red] (apache3.south) |- (shaFS.east);   
    
  % --- LEYENDA ---
  \matrix [draw, below =10cm of f5, column sep=0.5cm, row sep=0.3cm, nodes={anchor=west}]
  {
    \node[comp, scale=0.5] {}; & \node {Componentes}; &
    \node[lb, scale=0.5] {}; & \node {Load Balancer}; \\
    \node[server, scale=0.7] {}; & \node {Unix Server DEV+UAT+Prd}; &
    \node[server2, scale=0.7] {}; & \node {Unix Server +Prd}; \\
    \node[webserver, scale=0.5] {}; & \node {Web Server}; &
    \node[appserver, scale=0.5] {}; & \node {App Server}; \\
    \node[db, scale=0.6] {};    & \node {Bases de datos de proyectos}; &
    \node[sharedFS, scale=0.6] {};    & \node {Shared FS}; \\
  };    

\end{tikzpicture}
    \caption{Componentes de la arquitectura de la webapp}
    \label{fig:arq_webapp} 
\end{figure}

Finalmente y de acuerdo a los estándares tecnológicos de este proyecto, se utilizó el siguiente stack tecnológico:

    \begin{itemize}
        \item [Loadbalancer de F5]
        \item [Webserver: Apache]
        \item [Application server: Tomcat]
        \item [RDBMS: Oracle]
        \item [sistema operativo de servers: RH Linux]
    \end{itemize}
    
    \item \Gls{lux} (\ref{principios:leanUx})

    En el caso del diseño de la interfaz de usuario, no condiciona la arquitectura en sí de la aplicación, aunque influye en la selección de qué tecnologías se utilizarán para implementar las interacciones y los elementos de visualización. La interfaz de usuario, es el sistema de comunicación que tiene el producto de software para brindarle información sobre el estado del sistema, los contenidos y el resultado de las operaciones que el usuario realice con el mismo. 
    
    \item Tener una \gls{soa} (\ref{principios:soa})

    En este caso particular, se buscó implementar una arquitectura orientada a servicios, para la capa de negocio, de modo que pueda hacerse una separación, escalable de ser necesaria de ciertos servicios críticos y de alta demanda. de este modo, se puede seguir escalando la capacidad de la aplicación, en funcion de ciertas operaciones de alta demanda y críticas para brindar los servicios del portal. En otros casos, ciertos servicios se definieron como elementos reusables que pueden ser parte de un ecosistema más grande de aplciaciones y que tenía sentido independizar del bloque inicial de código, somo proyectos "hijos".
    
    \item Descentralizado y escalable (\ref{principios:federado})
    
    La posibilidad de descentralización, tiene que ver con poder distribuir tareas según responsabilidades de ciertos usuarios, en vez de tener un equipo centralizado de gente que opere y administre los contenidos y los manejos de accesos del portal. Esto implica que el volumen de usuarios con la capacidad de ejercer múltiples roles aumenta, en prácticamente un orden de magnitud, de 10x a 100x.
    
    La escalabilidad, se vuelve entonces un aspecto escencial para poder sostener un buen desempeño, desde el punto de vista de la performance y la experiencia de usuario, sin poner en riesgo la estabilidad del sistema. Para dicho fin, la arquitectura de \ref{fig:arq_webapp}, mencionada anteriormente se vuelve clave.
    
    \item Auditable (\ref{princpipios:auditable})
    explicar toda la lógica de auditoría en función de qué transacciones que hayan identificar como auditables y luego empezar a desarrollar cada uno, los temas de metadata que deben registrarse. en un log de auditoría se registra:

    \begin{itemize}
        \item userid: identificador unico del usuario que inicia la transacción.
        \item timestamp (formato de timestamp que representa el instante de tiempo en el que se hizo la transacción con nivel de granularidad de milisegundo).
        \item operación descripción de la operación que se realiza.
        \item campos afectados (con valores anteriores y nuevos, en caso de ediciones/actualizaciones).
        \item id de transacción (para el caso de que haya multiples operaciones en una transacción.
        \item duracion (en milisegundos).
        
    \end{itemize}

    Con toda esta metainformación, es posible obtener una vitácora de las transacciones ocurridas y poder realizar auditorías y análisis varios, tanto para evaluar estadísticas del sistema como así también patrones de uso y performance.
    
\end{enumerate}

\section{Arquitectura}
\section{Componentes principales}

Para avanzar con la implementación de la solución, se decidió implementar un patrón de \gls{mvc}. Dicho enfoque, permite separar claramente en capas lógicas la presentación e interfaz, la lógica de negocio y el modelo de datos, que se alinean muy bien con la arquitectura de webapp.

El mapeo del modelo MVC a la arquitectura de webapp fue hecha del siguiente modo:

Modelo (Model):

En el contexto de una webapp, el modelo representa la parte de la aplicación que maneja los datos y la lógica de negocio. Incluye las clases y objetos que interactúan con la base de datos para almacenar, recuperar y manipular datos.
El modelo se comunica directamente con el motor de bases de datos (RDBMS, Oracle, en este caso) para ejecutar consultas y obtener resultados.
En esta arquitectura de múltiples servidores, el modelo está distribuido, con diferentes instancias de aplicación (Tomcat) accediendo a la misma base de datos compartida a través de un sistema de archivos compartido (NFS).

Vista (View):

La vista es la parte de la aplicación que el usuario interactúa directamente. En una webapp, es la interfaz de usuario presentada en el navegador del usuario.
Las vistas se generan a partir del modelo y son dinámicas, adaptándose a los datos que el modelo proporciona. Las vistas pueden ser renderizadas en el servidor (Tomcat) y/o en el cliente (navegador).
En este caso, las vistas son gestionadas por el servidor web (Apache), que sirve archivos estáticos (HTML, CSS, JavaScript albergados en el NFS) y delega las solicitudes dinámicas al servidor de aplicaciones (Tomcat).

Controlador (Controller):

El controlador es el intermediario entre la vista y el modelo. Es responsable de manejar las solicitudes del usuario, interpretar las acciones y actualizar la vista y/o el modelo en consecuencia.
Aquí, el controlador se encarga de procesar las solicitudes HTTP, extraer los datos de la solicitud, invocar el código de negocio del modelo y devolver una vista actualizada al cliente.
El controlador está implementado en el servidor de aplicaciones, donde maneja las interacciones con el usuario y coordina el flujo de datos entre la vista y el modelo.

Arquitectura de la webapp:

En una arquitectura de webapp basada en MVC, los componentes se distribuyen a lo largo de los diferentes niveles de la stack tecnológico.
El servidor web (Apache) actúa como el punto de entrada para las solicitudes y atendiendo las solicitudes estáticas y delegando las dinámicas a los componentes tomcats.
El servidor de aplicaciones (Tomcat) ejecuta el controlador y el modelo, procesando las solicitudes y generando las vistas.
El sistema de archivos compartido proporciona el soporte al contenido estático que comparten las instancias de Apache y de Tomcats.
La base de datos (Oracle) proporciona almacenamiento persistente para los datos generados, mantenidos y accedidos por el modelo.

\begin{enumerate}
    \item \textbf{ReportsHub}:
        Codigo que contiene la lógica de negocios y la capa de presentación que se sirve a los browsers con los que se accede.
    \item \textbf{User Admin API}
    
    
    \item \textbf{RLS API}
    en este caso se puede explicar como funciona el API de roles para poder integrar como wrapper lo que se decidió implmentar.
    \item \textbf{RH DB} (base de datos de ReportsHub)
\end{enumerate}

Componentes del ecosistema:

\begin{enumerate}
    \item LDAP

    Explicación del directorio de LDAP y las técnicas de integración para poder meterse.
    
    \item herramientas de visualización
    Acá uno puede hablar de lo que son las herramientas de visualización y luego profundizar en variaciones.
    
    \item bases de datos de proyectos
    acpa se puede hablar del API

    
\end{enumerate}


\vspace{1cm}

\begin{figure}
    \centering
    \begin{tikzpicture}[node distance=0.5cm and 0.5cm]

% Nodos principales
\coordinate (anchor) at (-15cm,0cm);
\node[block, below=of anchor] (ldap){LDAP};
\node[block2, left=5cm of ldap] (api)  {Users API};
\node[block2, below left=1cm and 1cm of api] (reportshub) {ReportsHub};

% Herramientas BI
\node[block3, below right=1cm and -3cm of api] (qlik) {Qlik};
\node[block3, right=of qlik] (cognos) {Cognos};
\node[block3, right=of cognos] (spotfire) {Spotfire};
\node[block3, right=of spotfire] (powerbi) {PowerBI};

% Bases de datos
\node[db, below right=2cm and 0.1cm of qlik] (db1) {DB1};
\node[db, right=of db1] (db2) {DB2};
\node[db, right=of db2] (db3) {DB3};
\node[db2, left=of db1] (db4) {RH DB};

% API Row Level Security
\node[block2, below=4.7cm of reportshub] (rls) {RLS API};

% --- Conexiones ---
\draw[arrow] (reportshub.north) -- (api.west);
\draw[arrow] (reportshub.south) -- (db4.north);
\draw[arrow] (db4.north) -- (reportshub.south);
\draw[arrow] (api.west) -- (reportshub.north);

\draw[arrow] (api) -- (ldap);
\draw[arrow] (ldap) -- (api);


\draw[arrow] (ldap.south) -- (qlik.north);
\draw[arrow] (qlik.north) -- (ldap.south);
\draw[arrow] (ldap.south) -- (cognos.north);
\draw[arrow] (cognos.north) -- (ldap.south);
\draw[arrow] (ldap.south) -- (spotfire.north);
\draw[arrow] (spotfire.north) -- (ldap.south);
\draw[arrow] (ldap.south) -- (powerbi.north);
\draw[arrow] (powerbi.north) -- (ldap.south);

\draw[arrow] (qlik.south) -- (db1.north);
\draw[arrow] (qlik.south) -- (db2.north);
\draw[arrow] (db1.north) -- (qlik.south);
\draw[arrow] (db2.north) -- (qlik.south);

\draw[arrow] (cognos.south) -- (db2.north);
\draw[arrow] (db2.north) -- (cognos.south);

\draw[arrow] (spotfire.south) -- (db2.north);
\draw[arrow] (spotfire.south) -- (db3.north);
\draw[arrow] (db2.north) -- (spotfire.south);
\draw[arrow] (db3.north) -- (spotfire.south);

\draw[arrow] (powerbi.south) -- (db3.north);
\draw[arrow] (db3.north) -- (powerbi.south);

% Conexión RLS
\draw[arrow] (db1.south) -- (rls.east);
\draw[arrow] (db2.south) -- (rls.east);
\draw[arrow] (db3.south) -- (rls.east);
\draw[arrow] (rls.east) -- (db1.south);
\draw[arrow] (rls.east) -- (db2.south);
\draw[arrow] (rls.east) -- (db3.south);

\draw[arrow] (rls) -- (reportshub);
\draw[arrow] (reportshub) -- (rls);

  % --- LEYENDA ---
  \matrix [draw, below right= -1cm and 4cm of rls, column sep=0.5cm, row sep=0.3cm, nodes={anchor=west}]
  {
    \node[block2, scale=0.5] {}; & \node {Componentes de ReportsHub}; \\
    \node[block, scale=0.5] {}; & \node {Aplicaciones / Herramientas de BI}; \\
    \node[block3, scale=0.5] {}; & \node {Herramientas de visualización}; \\
    \node[db, scale=0.6] {};    & \node {Bases de datos de proyectos}; \\
  };
    \end{tikzpicture}
    \caption{Componentes de Arquitectura}
    \label{fig:arq_components2} 
\end{figure}

\section{Diseño de seguridad y Row Level Security (RLS)}
Por cuestiones de simplicidad y de diseño general, se trabajó con un esquema de seguirdad donde usuarios de ciertas áreas funcionales, pueden tener acceso a la información completa de dichas áreas, con limitaciones geográficas. en el caso de los usuarios "locales", en general se les otorga permiso vingulado al país donde operan. usuarios regionales, se les otorga visibilidad sobre la región (información consolidada a nivel regional) y del detalle de los paises incluidos en dicha región.
finalmente hay usuarios globales que tienen acceso a una capa global de datos (del area de interés del contenido publicado) y luego a los datos consolidados en niveles geográficos incluidos (regionales y paiese de cada región).

normlamnete para implementar el filtrado a nivel geográfico, se aplicaron técnicas de RLS, de modo que permite mediante el rol del usuario, poder tener scope a las filas (rows) que son relevantes a dicho usuario.

La implmentación, normalmente se aplica a nivel de modelo de datos, haciéndola inherente a las consultas, de modo que siempre se agregan automáticamente a todas las queries, where conditions limitando las filas con las que se trabaja por alcance geográfico.

Se puede agregar algún diagrama de RLS y algún modelo de datos (DER) que ayude a implmenentarlo.

\section{Flujos de datos e integraciones con sistemas existentes}

Workflows de aprobaciones y relación con los casos de uso. 

acá se puede trabajar las mejoras de los flujos de interacciones para cada caso de uso.


\section{Consideraciones de escalabilidad y resiliencia}

revisar si acá las condiciones de escalabilidad y resiliencia que se definieron en los principios de diseño se implementan en arquitectura.

\begin{tikzpicture}[
  node distance=1.5cm and 1cm,
  every node/.style={draw, fill=white, drop shadow, font=\sffamily, align=center},
  model/.style={ellipse, minimum width=2cm, minimum height=1cm},
  view/.style={rectangle, minimum width=3cm, minimum height=1.5cm},
  controller/.style={rectangle, minimum width=3cm, minimum height=1.5cm},
  browser/.style={rectangle, minimum width=2cm, minimum height=1cm},
  webserver/.style={rectangle, minimum width=2cm, minimum height=1cm},
  appserver/.style={rectangle, minimum width=2cm, minimum height=1cm},
  dbserver/.style={cylinder, shape border rotate=90, minimum width=1.5cm, minimum height=1.5cm},
  ap/.style={-Stealth, thick}
]

% Browser
\node[browser] (browser) {Browser};

% Web Server (Apache)
\node[webserver, below=of browser] (webserver) {Apache};

% App Server (Tomcat)
\node[appserver, below=of webserver] (appserver) {Tomcat};

% Database Server (Oracle)
\node[dbserver, below=of appserver] (dbserver) {Oracle DB};

% MVC Components
\node[controller, left=of appserver] (controller) {Controller};
\node[model, left=of controller] (model) {Model};
\node[view, left=of webserver] (view) {View};

% Arrows
\draw[ap] (browser) -- (webserver);
\draw[ap] (webserver) -- (controller);
\draw[ap] (controller) -- (model);
\draw[ap] (model) -- (dbserver);
\draw[ap] (dbserver) -- (model);
\draw[ap] (model) -- (controller);
\draw[ap] (controller) -- (view);
\draw[ap] (view) -- (webserver);
\draw[ap] (webserver) -- (browser);

\end{tikzpicture}



\chapter{desarrollo y despliegue}
\section{Enfoque metodológico (Waterfall/Agile)}

Metodología Waterfall
En la metodología Waterfall, las fases del desarrollo de software se llevan a cabo de manera secuencial. Aquí tienes cómo se organiza el equipo en esta metodología:

Product Owner: Responsable de definir los requerimientos y asegurarse de que el producto final cumpla con las expectativas del cliente.
Scrum Master: Este rol no existe en Waterfall, pero podría haber un jefe de proyecto que coordina las tareas y asegura que las fases se completen a tiempo.
Programadores de Front End: Se encargan de la interfaz de usuario una vez que se ha definido y aprobado el diseño.
Programadores de Back End y de Base de Datos: Trabajan en la lógica del servidor y la estructura de la base de datos después de la fase de diseño.
Testers: Realizan pruebas finales después de que el desarrollo esté completo.
Operadores: Se encargan del despliegue y mantenimiento del sistema una vez que ha sido probado y aprobado.
Metodología Agile
En la metodología Agile, las fases del desarrollo de software son iterativas e incrementales. El equipo trabaja en sprints cortos y frecuentes para entregar funcionalidades completas. Aquí es cómo se organiza el equipo:

Product Owner: Define y prioriza el backlog del producto, asegurándose de que el equipo trabaje en las tareas de mayor valor.
Scrum Master: Facilita el proceso Agile y remueve obstáculos para el equipo.
Programadores de Front End: Trabajan en la interfaz de usuario en cada sprint, colaborando estrechamente con los programadores de back end.
Programadores de Back End y de Base de Datos: Desarrollan la lógica del servidor y la estructura de la base de datos en cada sprint.
Testers: Realizan pruebas continuas durante cada sprint, asegurando que las nuevas funcionalidades funcionen correctamente.
Operadores: Empiezan a involucrarse en el proceso de despliegue más frecuentemente, aunque aún puede haber una separación entre desarrollo y operaciones.

\section{Planificación y backlog inicial}

\section{Definición de historias de usuario clave}

\section{Iteraciones y sprints principales}

\section{Validaciones con usuarios de negocio}

\section{Gestión de calidad}

\subsection{Pruebas funcionales}

\subsection{Pruebas de seguridad}
\subsection{Pruebas de performance}
\subsection{Pruebas de integración}
\section{Hitos alcanzados y ajustes sobre la marcha}
\section{Estrategia de despliegue y fases de rollout.}
\section{Piloto inicial y resultados}
\section{Escalamiento a más áreas y usuarios}
\section{Plan de comunicación y capacitación}
\section{Materiales de soporte y guías de usuario}
\section{Gestión del cambio organizacional}
\section{Mecanismos de soporte post-lanzamiento}


\chapter{evaluacion}

Medir el éxito de este proyecto fue fundamental por varias razones:

\begin{enumerate}
    \item Evaluación del rendimiento: Pudimos evaluar si los objetivos del proyecto se cumplieron y si los resultados esperados se alcanzaron. Esto ayudó a determinar si el proyecto fue efectivo y eficiente.
    \item Identificación de áreas de mejora: Al medir el éxito, tambien pudimos identificar áreas que funcionaron bien y las que necesitan mejoras. Esto fue vital para aprender de la experiencia y aplicar esos aprendizajes a futuros proyectos.
    \item Justificación de inversiones: Este proyecto requirió de una inversión, tanto en términos de tiempo como de dinero. La evaluación del éxito ayudó a justificar esta inversión, a demostrar el retorno sobre la inversión (ROI) a los interesados y sponsors y habilitar futuras fases de financiamiento para su evolución en el tiempo.
    \item Motivación y reconocimiento: El haber medido y celebrado el impacto del proyecto, motivó al equipo del proyecto, reconociendo sus esfuerzos y logros. Esto mejoró la moral e imagen del equipo en la compañía y ayudó a fomentar una cultura de éxito y colaboración.
    \item Mejora continua: La evaluación del éxito proporcionó datos y conocimiento que se utilizaron luego para mejorar los procesos y metodologías de gestión de proyectos. Esto contribuye a la mejora continua y a la optimización de futuras iniciativas.
    \item Rendición de cuentas: Permitió al product owner rendir cuentas a los sponsors y stakeholders y demostrar que gestionaron los recursos de manera adecuada y responsable, cumpliendo con los objetivos establecidos.
\end{enumerate}

A continuación, se detalla la metodología que se utilizó y los resultados obtenidos.

\section{Metodología de evaluación}

La metodología de evaluación se basó en tomar diferentes métricas y evaluarlas en función de las distintas fases de despliegue del proyecto.

Se evaluaron dichas métricas en un estado inicial (previo al despliegue del proyecto), como punto de partida, llamado \emph{baseline} y luego, se tomó una primer evaluación, a los 12 meses de haber iniciado (y desplegado el proyecto) y se re-evaluaron dichas métricas para poder ver la evolución.

Las métricas que se evaluaron son las detalladas inicilamnete en la Sec.\ref{success:metrics} y rondan en tres ejes principales:

\begin{enumerate}
    \item Métricas de experienca de usuario (\ref{success:metric_ux})
    \item Métricas de adopción de la herramienta (\ref{success:adoption}) 
    \item Métricas de ahorro y optimización (\ref{success:timesaving})
    \item Métricas de Auditoría y Cumplimiento (\ref{success:compliance})
\end{enumerate}

\section{Resultados cuantitativos}

A continuación, se detallan las mediciones para cada metrica comparando el punto de partida (línea base) y el resultado al cabo de 12 meses del primer despliegue a producción.

En primer lugar, se puede ver en la \emph{Tabla \ref{tab:tiempo_respuesta}}, que los tiempos de repuesta están en su mayoría alineados con lo planificado, aunque en el caso de la gestión de usuario, hasta ese momento no se había logrado alcanzar la meta propuesta.


\begin{table}[h!]
    \centering
    \begin{tabular}{lcccccc}
        \hline
        \textbf{ms por tx} & \textbf{Línea Base} & \textbf{12meses Prom} & \textbf{min} & \textbf{max} \\ \hline
        Gestion de Geografia \ref{usecases:geolevel1}& N/A & 750 & 750 & 1020\\ \hline
        Gestion de estructura 2 \ref{usecases:geolevel2}& N/A & 857 & 750 & 1020\\ \hline
        Gestion de usuarios \ref{usecases:useradmin}& N/A & 1123 & 750 & 1020\\ \hline
        Gestion de Acceso \ref{usecases:accessmgmnt} & N/A & 950 & 750 & 1020\\ \hline
        Navegacion \ref{usecases:browse}& N/A & 1050 & 750 & 1020\\ \hline
        Gestion Contenido \ref{usecases:contentaccess}& N/A & 600 & 750 & 1020\\ \hline
    \end{tabular}
    \caption{Tiempos de respuesta en ms por transacción}
    \label{tab:tiempo_respuesta}
\end{table}

Respecto de las métricas de adopción, se puede ver en la \emph{Tabla \ref{tab:adopcion_contenidos}}, que como el sistema es nuevo, se partió de una base de accesos, reportes, usuarios y países en 0 (cero), y al cabo de 12 meses se pudo evaluar el tráfico, en función de los contenidos y equipos que se fueron sumando. 

Respecto de los objetivos de cobertura en términos de reportes, el avance fue significativo, cubriendo en todos los casos, la mayoría de los reportes, usuarios y países existentes. Queda espacio para seguir expandiendo y es un punto que se abordará en el trabajo futuro propuesto, sobre el final de esta tesis.

\begin{table}[h!]
    \centering
    \begin{tabular}{lcccc}
        \hline
        \textbf{Métricas de Adopción} & \textbf{Línea Base} & \textbf{12meses} & \textbf{Objetivo} \\ \hline
        Accesos Mensuales & 0 & 7000(prom)/10000(max) & N/A \\ \hline
        Reportes & 0 & 307 & 550 \\ \hline
        Usuarios Únicos & 0 & 1785 & 2500 \\ \hline
        Países & 0 & 42 & 50 \\ \hline
    \end{tabular}
    \caption{Métricas de Adopción y Contenidos}
    \label{tab:adopcion_contenidos}
\end{table}

Finalmente, en la \emph{Tabla \ref{tab:ahorros_optimizacion}}, se hizo una evaluación de los tickets de soporte creados de modo automático y los pedidos de acceso resueltos. Si historicamente, la creación manual de un ticket de soporte, llevaba alrededor de 5 minutos, es fácil ver que solamente en tiempo de ingreso de datos, se ahorraron el equivalente a unas 86 horas de trabajo consolidadas, que aunque en período de 12 meses, podría ser un tiempo marginal, representa menos desperdicio y una mejor experiencia para los usuarios, con tareas que ya no exsiten de modo manual.

Respecto a los pedidos de acceso que se registraron, por cada pedido de acceso, el tiempo de resolución orignal, pasando por múltiples personas y equipos hasta que era finalmente rechazado o aprobado, se calculaba en unas 48hs hábiles (en promedio), con lo cual, en este caso, los ahorros de tiempo por la simplificación del proceso de gestión de accesos, representó un 95\% de ahorro por pedido, lo cual se traduce en casi 2750 días de espera eliminados.


\begin{table}[h!]
    \centering
    \begin{tabular}{lcccc}
        \hline
        \textbf{Métricas de Ahorro y Optimizacion} & \textbf{Línea Base} & \textbf{12meses} & \textbf{Objetivo} \\ \hline
        Tickets de Soporte Creados Automáticamente & 0 & 1036 & N/A \\ \hline
        Pedidos de Acceso & 0 & 1447 & N/A \\ \hline
    \end{tabular}
    \caption{Métricas de Ahorro y Optimizacion}
    \label{tab:ahorros_optimizacion}
\end{table}



\section{Resultados cualitativos}

Adicionalmente, se hizo una medición cualitativa del universo de usuarios, identificando aquellos de mayor peso en la organización y luego todo el resto. Esto es de fundamental relevancia, porque cuanto más alto en la escala de la organización se llega, mayor exposición se logra de la solución, y si bien esto puede representar un riesgo desde el punto de vista de posibles fallas de funcionamiento, ha sido muy bien recibido y ha generado aliados y promotores de la plataforma, con peso significativo en la organización. El detalle de los perfiles de los usuarios, se detalla a continuación, en la en la \emph{Tabla \ref{tab:perfiles_alto_impacto}}.

\begin{table}[h!]
    \centering
    \begin{tabular}{lccc}
        \hline
        \textbf{Perfiles de Usuarios de Alto Impacto} & \textbf{Línea Base} & \textbf{12meses} \\ \hline
        Vice Presidentes Sr & 0 & 2 \\ \hline
        Vice Presidentes & 0 & 2 \\ \hline
        Vice Presidentes Asociados & 0 & 2 \\ \hline
        Directores Ejecutivos & 0 & 5 \\ \hline
        Directores & 0 & 15 \\ \hline
        Otros & 0 & 1759 \\ \hline
    \end{tabular}
    \caption{Perfiles de Usuarios de Alto Impacto}
    \label{tab:perfiles_alto_impacto}
\end{table}

\section{Impacto en seguridad y cumplimiento}

Un objetivo clave del proyecto era abordar desafíos de cumplimiento que la organización enfrentaba debido a la falta de registros centralizados y mecanismos de auditoría eficientes. Para ello, el sistema de logs de auditoría automáticos y completos que permitió una captura detallada y precisa de todas las actividades relevantes.

El proyecto abordó estos desafíos mediante la implementación de un sistema de logs de auditoría automatizados, que incluía las siguientes características:
\begin{enumerate}
\item \emph{Captura Automática de Datos}: Todas las actividades relevantes se registraban automáticamente{,} eliminando la necesidad de intervención manual y garantizando la integridad de los datos.
\item \emph{Registros Completos y Detallados}: Los logs incluían información detallada sobre cada acción{,} incluyendo el usuario responsable{,} la fecha y hora{,} y la naturaleza de la acción.
\item \emph{Acceso y Monitoreo en Tiempo Real}: Los responsables de cumplimiento podían acceder a los registros en tiempo real{,} permitiendo una supervisión continua y proactiva.
\end{enumerate}

Beneficios Obtenidos:
La implementación del sistema de logs de auditoría completos y automatizados resultó en varios beneficios para la organización:
\begin{itemize}
    \item \emph{Mejora en el Cumplimiento}: La capacidad de mantener registros detallados y accesibles permitió a la organización cumplir con las regulaciones de manera más efectiva y reducir el riesgo de sanciones.
    \item \emph{Aumento de la Transparencia}: La disponibilidad de logs detallados mejoró la transparencia dentro de la organización, facilitando la rendición de cuentas y la toma de decisiones basada en datos.
    \item \emph{Eficiencia Operativa}: La automatización de la captura de datos y la reducción de auditorías manuales permitieron una asignación más eficiente de recursos y una mejora en la eficiencia operativa.
    \item \emph{Detección Temprana de Incidentes}: El acceso en tiempo real a los registros permitió la detección y respuesta rápida a incidentes, mejorando la seguridad y la gestión de riesgos.
\end{itemize}

El proyecto fue exitoso en resolver los desafíos de cumplimiento mediante la implementación de un sistema de logs de auditoría completos y automatizados. La organización no solo mejoró su capacidad de cumplir con las regulaciones, sino que también aumentó la transparencia, eficiencia operativa y capacidad de respuesta ante incidentes. Este enfoque proactivo y basado en datos ha fortalecido la posición de la organización en términos de cumplimiento y gestión de riesgos.

\section{Conclusiones de la evaluación}

El proyecto, luego de 12 meses, logró resultados destacados e impactos tangibles. siendo ahora nuevo estándard dentro de la corporación y permitiendo un nivel de simplicidad, acceso y eficiencia que parecia imposible. Una solución simple (aunque con cierta complejidad técnica) logró atender los desafíos propuestos en incluso ser adoptada globalmente.


% \chapter{Conclusiones}

% En esta sección, el objetivo es destacar el aporte de esta tesis como así también del proyecto en cuestión.

% \section{Resumen de los logros principales}

% Durante el desarrollo de esta tesis, se han alcanzado varios logros significativos en el proyecto ReportHub:

% \begin{enumerate}
    
%     \item Identificación de Problemas y Limitaciones: Se identificaron las principales limitaciones y problemas enfrentados por los usuarios y los equipos de datos en la organización, tales como la gestión de accesos y el manejo del portafolio de soluciones de inteligencia de negocios.
    
%     \item Desarrollo de la Propuesta: Se diseñó y desarrolló ReportHub, una aplicación web con una arquitectura orientada a servicios (SOA) y principios de diseño minimalistas y defensivos. Esta herramienta es escalable, auditable y centrada en las necesidades de la organización.
    
%     \item Implementación y Despliegue: Se implementó un enfoque metodológico robusto para el desarrollo y despliegue de ReportHub, incluyendo una estrategia de despliegue en fases y un plan de comunicación y capacitación eficaz.
    
%     \item Evaluación de Resultados: La evaluación del proyecto mostró resultados cuantitativos y cualitativos positivos que resaltaron la mejora en la gestión de contenidos y accesos, así como un impacto significativo en la seguridad y el cumplimiento.
% \end{enumerate}

% \section{Relevancia del proyecto para la organización}
% El proyecto ReportHub ha demostrado ser de gran relevancia para la organización al mejorar la eficiencia en la gestión de datos y contenidos. Esta herramienta ha proporcionado una solución integral a los problemas identificados, lo que ha resultado en una mejor toma de decisiones y una mayor agilidad en los procesos internos.

% \section{Impacto estratégico a largo plazo}
% A largo plazo, ReportHub tiene el potencial de transformar la forma en que la organización gestiona sus datos y contenidos. La herramienta contribuirá a una mayor transparencia y control sobre los datos, facilitando el cumplimiento de regulaciones y mejorando la seguridad. Además, su diseño escalable permite una fácil adaptación a futuras necesidades y expansiones.

% \section{Próximos pasos sugeridos (roadmap evolutivo)}
% Para continuar con el éxito de ReportHub, se sugieren los siguientes pasos:
% \begin{itemize}
%     \item Expansión de Funcionalidades**: Incorporar nuevas funcionalidades basadas en el feedback de los usuarios para mejorar aún más la experiencia y la utilidad de la herramienta.
%     \item Monitoreo y Mejora Continua**: Establecer un sistema de monitoreo continuo para evaluar el desempeño de ReportHub y realizar mejoras periódicas basadas en los resultados obtenidos.
%     \item Capacitación Continua**: Ofrecer capacitaciones regulares a los usuarios para asegurar un uso óptimo de la herramienta y maximizar su impacto positivo en la organización.
% \end{itemize}
    

% \newpage

\chapter{Conclusiones de esta tesis}

\section{Resumen de los logros principales}

Esta tesis, resume, el recorrido completo para implementar una solución, a un problema que era recurrente y no estaba resuelto de ningún modo razonable. Desde la conceptualización del problema y su formalización, hasta el desarrollo, implementación y despliegue productivo, midiéndo el impacto en el entorno donde se desplegó. Los logros que documenta la tesis, sobre este proyecto, están descriptos a continuación:

\begin{itemize}
    \item Identificación de Problemas y Limitaciones: 
    
El primer logro significativo fue la identificación de las principales limitaciones y problemas enfrentados por los usuarios y los equipos de datos en la organización. Este paso inicial fue fundamental para definir el alcance y los objetivos del proyecto. Mediante un exhaustivo análisis de la situación inicial, se detectaron problemas críticos como la gestión de accesos, el consumo de contenidos, y el manejo del portafolio de soluciones de inteligencia de negocios. Estas limitaciones representaban obstáculos importantes para la eficiencia operativa y la toma de decisiones en la organización. La identificación de estos problemas permitió establecer una base sólida para el desarrollo de soluciones efectivas.

    \item Desarrollo de la Propuesta:

El siguiente logro fue el diseño y desarrollo de ReportHub, una aplicación web innovadora (para el entorno corporativo en cuestión) con una arquitectura orientada a servicios (SOA) y principios de diseño minimalistas y defensivos. ReportHub se concibió como una herramienta escalable y auditable, centrada en satisfacer las necesidades específicas de la organización. La aplicación incluyó funcionalidades clave como la gestión de usuarios, la administración de contenidos, y la navegación de catálogos, entre otras. El desarrollo de ReportHub implicó un enfoque colaborativo, involucrando a diversas partes interesadas para asegurar que la solución final fuera robusta y alineada con los objetivos estratégicos de la organización.

    \item Implementación y Despliegue:

La implementación de ReportHub se llevó a cabo mediante un enfoque metodológico robusto que incluyó una estrategia de despliegue en fases. La primera fase, conocida como el Producto Mínimo Viable (MVP), se realizó con un grupo piloto para probar y refinar la herramienta antes de su expansión a más áreas y usuarios. 

Esta estrategia permitió identificar y solucionar problemas tempranos, asegurando una transición suave hacia la fase de expansión. Además, se desarrolló un plan de comunicación y capacitación eficaz para garantizar que todos los usuarios estuvieran bien informados y capacitados en el uso de ReportHub. Esta fase también incluyó la creación de mecanismos de soporte post-lanzamiento para atender cualquier necesidad o problema que pudiera surgir después de la implementación.

    \item Evaluación de Resultados:

Finalmente, la evaluación del proyecto mostró resultados cuantitativos y cualitativos positivos. Los datos recopilados durante la evaluación indicaron mejoras significativas en la gestión de contenidos y accesos, así como un impacto notable simplificando y automatizando la seguridad y el cumplimiento de normativas. 

Los usuarios reportaron una mayor eficiencia en sus tareas diarias y una mejor experiencia general con la nueva herramienta. Estos resultados no solo validaron las hipótesis iniciales del proyecto, sino que también destacaron la efectividad de ReportHub como una solución integral para los desafíos identificados.

\end{itemize}
En resumen, los logros alcanzados durante el desarrollo de esta tesis han sido numerosos y significativos. La identificación de problemas y limitaciones, el desarrollo de una propuesta innovadora, la implementación y despliegue efectivos, y la evaluación positiva de resultados, han contribuido a mejorar la eficiencia operativa y la toma de decisiones en la organización. Estos logros subrayan la importancia del proyecto ReportHub y su impacto positivo en la organización.

\section{Relevancia del proyecto para la organización}

El proyecto ReportHub ha demostrado ser de gran relevancia para la organización por varias razones. En primer lugar, ha mejorado significativamente la eficiencia en la gestión de datos y contenidos. Antes de la implementación de ReportHub, los usuarios enfrentaban múltiples desafíos relacionados con la gestión de accesos y el manejo de contenidos. Estos problemas no solo ralentizaban los procesos internos, sino que también afectaban la calidad de las decisiones tomadas. Con la introducción de ReportHub, estos desafíos se han abordado de manera efectiva, permitiendo una gestión más fluida y eficiente de los datos.


La importancia de ReportHub también se refleja en su capacidad para proporcionar una solución integral a los problemas identificados. La herramienta ha sido diseñada para ser escalable y adaptable, lo que significa que puede evolucionar con las necesidades cambiantes de la organización. Esta flexibilidad es crucial en un entorno empresarial dinámico, donde las demandas y los requisitos pueden cambiar rápidamente. ReportHub ha demostrado ser capaz de adaptarse a estas demandas, asegurando que la organización siempre tenga acceso a una herramienta eficaz y actualizada.


Además, ReportHub ha mejorado la transparencia y el control sobre los datos. La capacidad de auditar y monitorear las actividades dentro de la plataforma ha permitido a la organización mantener un control más estricto sobre el uso y la gestión de los contenidos/reportes. Esto no solo ha mejorado la seguridad y el cumplimiento de normativas, sino que también ha aumentado la confianza de los usuarios en la herramienta. Saber que sus actividades están siendo monitoreadas de manera efectiva ha llevado a una mayor adherencia a las políticas y procedimientos establecidos.


Otro punto a destacar, es que la experiencia del usuario ha sido mejorada de modo consitente y de inicio a fin, de manera completa.ReportHub ha permitido a los usuarios acceder a la información que necesitan de manera rápida y eficiente. 


En términos de impacto estratégico, ReportHub ha establecido una base sólida para futuras iniciativas de gestión de datos. La implementación exitosa de esta herramienta ha demostrado que la organización está comprometida con la mejora continua y la innovación y a su vez cuenta con herramientas que permiten una gestión, basada en datos, del portafolio de productos de BI y futuras inversiones.


En resumen, la relevancia del proyecto ReportHub para la organización es innegable. Ha mejorado la eficiencia operativa, la transparencia y el control sobre los reportes/contenidos de BI. Estos beneficios han tenido un impacto positivo en la organización, posicionándola de caras a demandas crecientes de gestión efectiva y eficiente de sus datos y procesos.

\section{Impacto estratégico a largo plazo}

A largo plazo, ReportHub tiene el potencial de transformar la forma en que la organización gestiona sus datos y contenidos. Esta transformación no solo se limita a la mejora de procesos internos, sino que también tiene implicaciones estratégicas de gran alcance. A continuación se detallan los aspectos clave del impacto estratégico de ReportHub a largo plazo:

\begin{itemize}
    \item Transparencia y Control:

Una de las principales ventajas estratégicas de ReportHub es su capacidad para proporcionar una mayor transparencia y control sobre los datos. En un entorno regulado, donde el cumplimiento de normativas es crucial, tener una herramienta que permita auditar y monitorear las actividades es esencial. ReportHub facilita el cumplimiento de regulaciones al proporcionar un registro claro y accesible de todas las actividades. Esta transparencia no solo ayuda a evitar sanciones y penalidades, sino que también mejora la reputación de la organización ante reguladores y stakeholders externos.

    \item Adaptabilidad y Escalabilidad:

El diseño escalable y adaptable de ReportHub permite que la herramienta evolucione con las necesidades de la organización. A medida que la organización crece y se enfrenta a nuevos desafíos, ReportHub puede adaptarse para seguir siendo una herramienta útil y relevante. Esta capacidad de adaptación es crucial para mantener la competitividad en un entorno empresarial en constante cambio. Además, la escalabilidad de ReportHub significa que puede ser implementada en diferentes áreas y departamentos de la organización, ampliando su impacto y beneficios.

    \item Mejora Continua:

La capacidad de ReportHub para facilitar la mejora continua es otro aspecto clave de su impacto estratégico. La herramienta permite un monitoreo constante del rendimiento y el uso, lo que facilita la identificación de áreas de mejora. Esta capacidad de monitoreo y ajuste continuo asegura que la herramienta siempre esté optimizada para ofrecer el mejor rendimiento posible. Además, la recopilación de feedback de los usuarios permite realizar mejoras basadas en las necesidades reales y cambiantes de los usuarios, asegurando que ReportHub siga siendo una herramienta valiosa a largo plazo.

    \item Seguridad y Cumplimiento:

La seguridad de los datos es una preocupación creciente en el entorno empresarial actual. ReportHub ha sido diseñada con un enfoque robusto en la seguridad, incluyendo características como la seguridad a nivel de fila (Row Level Security, RLS) y la capacidad de auditar y monitorear todas las actividades. Estas características aseguran que los datos estén protegidos contra accesos no autorizados y que cualquier actividad inusual sea detectada y abordada de manera oportuna. La capacidad de cumplir con las normativas de seguridad y protección de datos es una ventaja estratégica importante, ya que ayuda a evitar sanciones y a mantener la confianza de los stakeholders.

\end{itemize}

En resumen, el impacto estratégico de ReportHub a largo plazo es significativo. La herramienta proporciona transparencia y control, es adaptable y escalable, facilita la mejora continua, asegura la protección de datos y cumplimiento normativo, y posiciona a la organización como innovadora y competitiva. Estos beneficios estratégicos aseguran que ReportHub seguirá siendo una herramienta valiosa y relevante para la organización en el futuro.

\section{Próximos pasos sugeridos (roadmap evolutivo)}

Para asegurar el éxito continuo de ReportHub y maximizar su impacto positivo en la organización, se sugieren los siguientes pasos como parte de un roadmap evolutivo:

\begin{itemize}
    \item Expansión de Funcionalidades:
    
Uno de los primeros pasos sugeridos es la expansión de las funcionalidades de ReportHub. Basándose en el feedback de los usuarios, se pueden identificar áreas donde la herramienta puede mejorar aún más su utilidad. Por ejemplo, se podrían incorporar funcionalidades avanzadas de análisis de datos, integración con otras herramientas de gestión empresarial, y mejoras en la interfaz de usuario para facilitar una experiencia más intuitiva. La expansión de funcionalidades no solo mejorará la experiencia del usuario, sino que también aumentará el valor de la herramienta para la organización.

    \item Monitoreo y Mejora Continua: 

Establecer un sistema de monitoreo continuo es crucial para asegurar que ReportHub siga funcionando de manera óptima. Este sistema debe incluir la recopilación de datos sobre el rendimiento de la herramienta, así como el feedback de los usuarios. Con esta información, se pueden realizar ajustes y mejoras periódicas para mantener la herramienta actualizada y relevante. Además, el monitoreo continuo permite identificar de manera proactiva cualquier problema o área de mejora, asegurando que la herramienta siempre esté optimizada para ofrecer el mejor rendimiento posible.

    \item Capacitación Continua:

La capacitación continua de los usuarios es otro paso importante en el roadmap evolutivo de ReportHub. Ofrecer capacitaciones regulares asegura que los usuarios estén siempre informados sobre las nuevas funcionalidades y mejoras de la herramienta. Además, la capacitación ayuda a maximizar el impacto positivo de la herramienta al asegurar que los usuarios sepan cómo utilizarla de manera efectiva. Se pueden organizar talleres, seminarios web y sesiones de formación en línea para mantener a los usuarios actualizados y comprometidos.

    \item Expansión al resto de los equipos de BI:
    
De las métricas presentadas, se deduce que todavía queda trabajo por hacer para lograr una cobertura completa de los contenidos y reportes existentes en el mundo de BI de la compañía. En función de los beneficios logrados hasta el momento, se evalúa continuar hasta lograr una cobertura total.

\end{itemize}


En resumen, los próximos pasos sugeridos para el roadmap evolutivo de ReportHub incluyen la expansión de funcionalidades, el monitoreo y mejora continua, y la expansión a nuevos equipos de BI, para lograr una cobertura completa de la gestión de las soluciones existentes. Estos pasos asegurarán que ReportHub siga siendo una herramienta valiosa y relevante para la organización, maximizando su impacto positivo a largo plazo.


\chapter{Lecciones aprendidas}

\section{Aspectos que funcionaron bien}

Durante el desarrollo y la implementación del proyecto ReportHub, se identificaron varios aspectos que funcionaron de manera efectiva. En primer lugar, la colaboración entre los diferentes equipos de la organización fue uno de los mayores éxitos. La comunicación fluida y el trabajo en equipo permitieron resolver problemas de manera rápida y eficiente, asegurando que el proyecto avanzara según lo planeado. Además, la metodología ágil utilizada permitió una entrega iterativa y continua, lo que facilitó la incorporación de \textit{feedback} y la realización de ajustes en cada fase del proyecto. La arquitectura orientada a servicios (SOA) también demostró ser una elección acertada, ya que permitió una escalabilidad y flexibilidad significativas en el desarrollo y la implementación de la solución.

\section{Dificultades y cómo se superaron}

A lo largo del proyecto, se enfrentaron varias dificultades que requirieron soluciones creativas y eficaces. Una de las principales dificultades fue la evolución de la metodología de desarrollo, donde quedó claro, que con el enfoque inicial, no se hubiese podido avanzar ni con la velocidad, ni con la calidad necesaria. 

\section{Factores clave de éxito}

Varios factores clave contribuyeron al éxito del proyecto ReportHub. En primer lugar, el compromiso y la dedicación del equipo de proyecto fueron fundamentales. Desde el inicio, todos los miembros del equipo estuvieron alineados con los objetivos del proyecto y trabajaron de manera colaborativa para alcanzarlos. La metodología ágil también jugó un papel crucial, permitiendo una entrega iterativa y la incorporación de \textit{feedback} continuo. Además, la elección de una arquitectura escalable y auditable permitió que la solución se adaptara a las necesidades cambiantes de la organización y facilitó el cumplimiento de normativas y la mejora de la seguridad. Por último, el apoyo y la participación de los \textit{stakeholders} fueron vitales para asegurar la alineación del proyecto con los objetivos estratégicos de la organización.

\section{Aspectos a mejorar}

A pesar de los éxitos alcanzados, se identificaron varios aspectos que podrían mejorarse en futuros proyectos. Uno de estos aspectos es la gestión del cambio. Aunque se llevaron a cabo sesiones de capacitación y comunicación, algunos usuarios aún mostraron resistencia al cambio. En futuros proyectos, se podría considerar la implementación de un plan de gestión del cambio más robusto, que incluya una comunicación más temprana y continua, así como una mayor participación de los usuarios en las etapas iniciales del proyecto. Otro aspecto a mejorar es la documentación. Aunque se creó documentación detallada, algunos usuarios encontraron que esta no era lo suficientemente clara o accesible. Mejorar la claridad y la accesibilidad de la documentación podría facilitar una adopción más rápida y eficiente de la nueva herramienta.

\section{Aprendizajes organizacionales}

El proyecto ReportHub proporcionó varios aprendizajes valiosos a nivel organizacional. Uno de los principales aprendizajes fue la importancia de la colaboración y la comunicación efectiva entre los diferentes equipos y departamentos. Este proyecto demostró que el trabajo en equipo y la comunicación fluida son esenciales para el éxito de cualquier iniciativa. Además, se aprendió la importancia de la flexibilidad y la adaptabilidad. La metodología ágil y la arquitectura escalable permitieron que el proyecto se adaptara a los cambios y necesidades emergentes de la organización, lo que fue crucial para su éxito. También se destacó la importancia de involucrar a los \textit{stakeholders} desde el inicio del proyecto para asegurar su alineación con los objetivos estratégicos de la organización.

\section{Recomendaciones para futuros proyectos}

Basándonos en las lecciones aprendidas del proyecto ReportHub, se pueden hacer varias recomendaciones para futuros proyectos. En primer lugar, es esencial implementar un plan de gestión del cambio robusto que incluya una comunicación temprana y continua, así como la participación de los usuarios en las etapas iniciales del proyecto. También se recomienda mejorar la claridad y la accesibilidad de la documentación para facilitar una adopción más rápida y eficiente de las nuevas herramientas. Además, es importante mantener un enfoque colaborativo y asegurar una comunicación efectiva entre los diferentes equipos y departamentos de la organización. Finalmente, se sugiere seguir utilizando metodologías ágiles y arquitecturas escalables para asegurar la flexibilidad y la adaptabilidad de los proyectos a las necesidades cambiantes de la organización.

%% ...

%%%% BIBLIOGRAFIA
\backmatter
%\bibliography{tesis}
\printbibliography
\printglossary



\end{document}
