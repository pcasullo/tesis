%\begin{center}
%\large \bf \runtitle
%\end{center}
%\vspace{1cm}
\chapter*{\runtitle}

\noindent In large multinational corporations, multiple business intelligence technologies and practices coexist, spanning the entire technology stack, from extraction, transformation, and loading (ETL) processes \cite{inmon1992} to visualization and analytics tools. Despite attempts to establish technological and governance standards \cite{isaca2012}, the decentralization of initiatives across functional areas and geographic levels creates a complex, heterogeneous ecosystem that is difficult to harmonize. This scenario poses significant challenges both in integrating solutions and in measuring their business impact, where the relationship between adoption and impact is often unclear \cite{davenport2006}. In this context, defining adoption metrics becomes critical to coherently managing the business intelligence solutions portfolio, although its implementation requires overcoming methodological and organizational barriers. Likewise, the user experience is affected by the lack of consistency in access and services, reinforcing the need to establish more integrated and standardized approaches in the management of these tools.
These corporations typically have practices for defining technological standards, which establish official tools or technologies for certain products and vendors, with the aim of simplifying and achieving technological synergies among selected tools and reducing combinations according to different use cases \cite{jacobson1986}.

\bigskip

\noindent\textbf{Keywords:} Data Warehouse, Business Intelligence, Portfolio Management, Automation, Distributed Access management, Global Process.