%\begin{center}
%\large \bf \runtitle
%\end{center}
%\vspace{1cm}
\chapter*{\runtitle}

\noindent 

Multinational corporations typically operate with highly heterogeneous Business Intelligence (\gls{bi}) ecosystems, where multiple technologies coexist across the \gls{techstack}, from \gls{dw} architectures and \gls{etl} processes \cite{inmon1992} to diverse analytical and visualization tools. Despite ongoing efforts to establish technological standards and governance frameworks \cite{isaca2012}, functional and geographical decentralization makes it difficult to harmonize practices and limits the ability to measure the real impact of analytics, particularly when the relationship between adoption and business value is not evident \cite{davenport2006}.

This thesis addresses this gap through the design and implementation of an integrated platform aimed at unifying the management of corporate analytical reports, standardizing the access experience, and enabling consistent measurement of adoption-related \glspl{kpi}. The proposed solution introduces a governance model that centralizes metadata, consolidates classification criteria, and formalizes request and approval workflows. It also incorporates \gls{ux} principles and portfolio management practices to improve traceability, reduce report duplication, and enhance the discoverability of analytical assets. The architecture includes mechanisms that integrate existing security rules and strengthen organizational control processes. Results demonstrate improvements in ecosystem visibility, operational consistency, and analytical capability to assess the effective usage of \gls{bi} solutions, contributing to a more coherent and optimized corporate analytics landscape.


\bigskip

Palabras clave: \emph{\gls{bi}, \gls{dw}, \Gls{techstack}, \Gls{ux}, \Gls{pm}, \Gls{auto}, \Gls{dam}, \Gls{gproc}, \Gls{gob}.}